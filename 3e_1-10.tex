\documentclass[12pt]{article}
\usepackage[utf8]{inputenc}
\usepackage[french]{babel}
\usepackage{amsmath,amssymb}

\newenvironment{exercice}[1]{\paragraph{Exercice #1.}}{}

\begin{document}
Neyl Gasmi
\begin{exercice}{1}
Calculer puis donner les résultats sous forme de fraction irréductible.\\
$A=\dfrac{1}{3}+\dfrac{2}{5}\times\dfrac{3}{4}$,\quad
$B=\left(\dfrac{1}{3}+\dfrac{2}{5}\right)\times\dfrac{3}{4}$,\quad
$C=\left(\dfrac{1}{3}+\dfrac{2}{5}\right)\div\dfrac{3}{4}$,\\
$D=\dfrac{4}{7}-\dfrac{1}{7}\times\dfrac{5}{3}$,\quad
$E=\dfrac{3}{7}-\dfrac{2}{5}\times\dfrac{15}{4}$,\quad
$F=\dfrac{\dfrac{3}{5}+\dfrac{2}{3}}{\dfrac{9}{4}+1}$. 
\end{exercice}

\begin{exercice}{2}
Calculer puis donner les résultats sous forme de fraction irréductible.\\
$A=\left(\dfrac{1}{5}-\dfrac{2}{4}\right)\times\left(\dfrac{3}{7}-\dfrac{1}{2}\right)$,\quad
$B=\left(\dfrac{3}{7}-\dfrac{1}{5}\right)\div\left(\dfrac{3}{2}-\dfrac{5}{4}\right)$,\\
$C=\dfrac{4}{3}-\dfrac{1}{3}\times\left(3+\dfrac{1}{2}\right)$,\quad
$D=\left(\dfrac{1}{4}-\dfrac{1}{3}\right)\times\left(\dfrac{3}{4}-\dfrac{3}{2}\right)$,\\
$E=\left(1-\dfrac{2}{3}\right)\div\left(1+\dfrac{1}{3}\right)$.
\end{exercice} 

\begin{exercice}{3}
Calculer les expressions suivantes lorsque $a=\dfrac{2}{3}$, $b=-\dfrac{3}{2}$ et $c=-\dfrac{3}{4}$.\\
$A=3a-b-c$,\quad
$B=-2a+4b-5c$,\quad
$C=6b^{2}-3a+5$,\quad
$D=\dfrac{1}{a}+\dfrac{1}{b}+\dfrac{1}{c}$,\quad
$E=\dfrac{a+c}{a-b}$.
\end{exercice} 

\begin{exercice}{4}
Calculer la valeur de $F=\dfrac{x+5y}{x}$ lorsque :\\
1) $x=\dfrac{2}{3}$ et $y=-4$ ;\quad
2) $x=-4$ et $y=-\dfrac{8}{5}$ ;\quad
3) $x=-\dfrac{1}{2}$ et $y=\dfrac{7}{10}$ ;\quad
4) $x=-\dfrac{2}{3}$ et $y=\dfrac{2}{15}$.
\end{exercice} 

\begin{exercice}{5}
Calculer puis donner les résultats sous forme de fraction irréductible.\\
$A=\dfrac{\dfrac{2}{3}+\dfrac{5}{7}}{\dfrac{2}{3}\times\dfrac{5}{7}}$,\quad
$B=\dfrac{5+\dfrac{3}{4}-\dfrac{1}{3}}{\,5-\dfrac{3}{4}+\dfrac{1}{3}\,}$,\quad
$C=\dfrac{\dfrac{1}{5}-\dfrac{3}{4}\times\dfrac{2}{3}}{\left(\dfrac{1}{5}-\dfrac{3}{4}\right)\times\dfrac{2}{3}}$.
\end{exercice}

\begin{exercice}{6}
Quel est le nombre qu’il faut ajouter au numérateur et au dénominateur de la fraction $\dfrac{5}{8}$ pour que la nouvelle fraction soit égale à $4$ ?
\end{exercice}

\begin{exercice}{7}
Trouver le nombre caché à la place de $\spadesuit$ et de $\clubsuit$.\\
1)\; $\dfrac{87}{60}=\dfrac{1}{2}+\dfrac{1}{4}+\dfrac{1}{3}+\dfrac{1}{6}+\dfrac{1}{\spadesuit}$\\
2)\; $\dfrac{31}{17+\dfrac{101}{\,8-\dfrac{7}{\clubsuit}\,}}=\dfrac{2015}{2014}$
\end{exercice} 

\begin{exercice}{8} \textit{(Puissances — Formules)}\\
\textbf{Série 1 —} Écrire les nombres sous la forme $3^{n}$ avec $n$ entier relatif.\\
$A=\dfrac{3^{5}\times 3^{2}}{3^{-7}}$,\;
$B=(3^{2}\times 3^{3})^{4}$,\;
$C=3^{2}\times (3^{3})^{4}$,\;
$D=\dfrac{\big(({-}3)^{2}\times 3^{2}\big)^{3}}{(-3)^{5}}$,\;
$E=\dfrac{\big(({-}3)^{2}\big)^{3}}{(-3)^{3}\times(-3)}$,\;
$F=\dfrac{3^{-2}\times 9^{-8}}{3^{4}\times 27^{-17}}$,\;
$G=\left(\dfrac{1}{3^{5}}\times (3^{2})^{3}\right)^{2}$,\;
$H=\dfrac{3^{2}\times 27}{81^{2}}$.\\[0.4em]
\textbf{Série 2 —} Écrire sous la forme $a^{n}$ avec $a$ entier naturel et $n$ entier relatif.\\
$A=2^{4}\times 4^{-5}$,\;
$B=2^{5}\times 8^{-3}$,\;
$C=\dfrac{8^{3}}{4^{3}}$,\;
$D=0{,}25^{-6}\times 4^{-25}$,\;
$E=5^{4}\times 25^{-7}\times 125^{2}$,\;
$F=\dfrac{7^{6}\times(-49)^{5}}{7^{-9}}$.\\[0.4em]
\textbf{Série 3 —} Écrire sous la forme $2^{n}\times 5^{m}$ avec $n,m$ entiers relatifs.\\
$A=\dfrac{2^{4}}{(2^{2}\times 5)^{5}}$,\;
$B=\dfrac{2\times (5^{2})^{3}}{2^{-3}}$,\;
$C=\dfrac{\big(2^{3}\times 2^{-4}\big)^{2}}{(5^{3})^{2}\times 5^{-5}}$,\;
$D=\dfrac{(10^{2})^{3}}{2^{-4}\times (25)^{6}}$,\;
$E=\left(\dfrac{2}{5}\right)^{4}\times\left(\dfrac{5^{2}}{2}\right)^{3}$,\;
$F=\dfrac{64^{3}\times 125^{4}}{250^{7}}$.
\end{exercice} 

\begin{exercice}{9}
\textit{Nombre de chiffres.} Déterminer le nombre de chiffres de $4^{16}\times 5^{25}$.
\end{exercice} 

\begin{exercice}{10}
\textit{Somme des chiffres.} Déterminer la somme des chiffres du nombre $10^{2046}-2046$.
\end{exercice} 
\end{document}
