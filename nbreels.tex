\documentclass[11pt,a4paper]{article}

\usepackage[T1]{fontenc}
\usepackage[utf8]{inputenc}
\usepackage[french]{babel}
\usepackage{amsmath,amssymb}
\usepackage[margin=2.5cm]{geometry}
\setlength{\parindent}{0pt}
\setlength{\parskip}{6pt}

\begin{document}

\begin{center}
    {\Large \texit{Neyl Gasmi}} \\[0.2em]
\end{center}

\section*{Exercice 1 \;(\,Équations faciles à résoudre\,)}
\begin{enumerate}
  \item Rappeler pourquoi le carré d'un nombre réel est toujours un nombre positif.
  \item Résoudre l'équation $x^2+1=0$ d'inconnue $x\in\mathbb{R}$.
  \item Vérifier que pour tout réel $x$, $(2x+1)^2=4x^2+4x+1$.\\
        Résoudre l'équation $4x^2+4x+3=0$ d'inconnue $x\in\mathbb{R}$.
  \item Résoudre l'équation $x^6+2x^2+7=0$ d'inconnue $x\in\mathbb{R}$.
\end{enumerate}

\section*{Exercice 2 \;(\,Équation se ramenant à l'égalité de deux carrés\,)}
Résoudre les équations suivantes d'inconnue $x\in\mathbb{R}$ :
\begin{enumerate}
  \item $(x-3)^2=(2x+5)^2$.
  \item $4(x+1)^2=9x^2$.
  \item $(3x-1)^2=x^2+x+\dfrac14$.
\end{enumerate}

\section*{Exercice 3}
\begin{enumerate}
  \item Écrire sous la forme d'un intervalle les réels $x$ vérifiant :
  \[
  3x-1\ge 0
  \quad \text{et}\quad
  2x+3>0.
  \]

  \item Résoudre les systèmes suivants (réponse sous forme d’intervalle) :
  \[
  \left\{
  \begin{aligned}
  x-3 &\le 0\\
  5+2x &\ge 0
  \end{aligned}
  \right.
  \qquad
  \left\{
  \begin{aligned}
  2x-1 &\le 0\\
  3-x &\le 0
  \end{aligned}
  \right.
  \qquad
  \left\{
  \begin{aligned}
  7x+1 &\ge 0\\
  2-3x &\ge 0
  \end{aligned}
  \right.
  \]
\end{enumerate}

\section*{Exercice 4 \;(\,Identités remarquables\,)}
Calculer
\[
A=\Bigl((\sqrt2+1)^{2017}+(\sqrt2-1)^{2017}\Bigr)^2
-\Bigl((\sqrt2+1)^{2017}-(\sqrt2-1)^{2017}\Bigr)^2.
\]

\medskip
Les exercices suivants ont été empruntés au site \texttt{http://pharedesmaths.free.fr}.

\section*{Exercice 5}
Sachant que $a,b,c$ vérifient $ab=0$, $a+b=4$ et $bc=12$, déterminer $c$.

\section*{Exercice 6}
$a,b,c,d$ vérifient $ab=8$, $bc=0$ et $a+b+c+d=15$.\\
Calculer $\bigl[(a+d)^3-b^2\bigr]\times c$.

\section*{Exercice 7}
Existe-t-il $a,b,c,d$ tels que $abcd=0$ et $\dfrac{ad}{bc}=2$ ?

\section*{Exercice 8}
$a,b,c,d$ vérifient
\[
(a-b)(b-c)(c-d)(d-a)=2(a-c).
\]
Sachant que deux et seulement deux sont égaux, les déterminer et justifier que $a>c$.

\section*{Exercice 9}
Déterminer $a,b,c,d$ sachant que
\[
(a^2+c^2)(b^2-d^2)=36,\quad (d^2-a^2)(c+b)=18,\quad
\frac{c^2+d^2}{a^2-b^2}=0.
\]

\section*{Exercice 10}
Résoudre $y=x^2+x-4$ et $(y+4)(y-x)=0$.

\section*{Exercice 11}
Résoudre
\[
\frac{x^2-y^2}{x+y}=0
\quad\text{et}\quad
(y-4)(x-3)=12.
\]

\section*{Exercice 12}
Résoudre $(y+3)(y-1)=0$ et $(y^2-y)(4-x)=12$.

\section*{Exercice 13}
$a,b,c,d$ vérifient
\[
(d+2)(a-6)=12,\ (a+d)(b-c)=0,\ (a^2+1)(d-2)=0,\ bc=a.
\]
Calculer $a+b+c+d$.

\section*{Exercice 14}
$a,b,c,d$ vérifient
\[
(a^2+d^2)(b-c)=0,\ bc=a,\ ad=18,\ abcd=162.
\]
Calculer $a+b+c+d$.

\section*{Exercice 15}
\begin{enumerate}
  \item Montrer que si $a^2=b^2$ alors $a=b$ ou $a=-b$.
  \item Si $x^2=y^2=z^2$, calculer $(x-y)(y-z)(z-x)$.
\end{enumerate}

\section*{Exercice 16}
Existe-t-il deux nombres non nuls vérifiant les conditions données sur produits et sommes ?

\section*{Exercice 17}
Sachant que $xy-(x+y)=-1$, développer $(x-1)(y-1)$ et conclure.

\section*{Exercice 18}
$a,b>0$ et $\dfrac{a}{1+b}=\dfrac{b}{1+a}$. Que conclure ?

\section*{Exercice 19}
\begin{enumerate}
  \item Vérifier $\dfrac{7^3+4^3}{7^3+3^3}=\dfrac{7+4}{7+3}$.
  \item Condition pour $\dfrac{a^3+b^3}{a^3+c^3}=\dfrac{a+b}{a+c}$.
\end{enumerate}

\section*{Exercice 20}
Montrer que
\[
A=\frac{(2\sqrt5+\sqrt7)^2-(2\sqrt5-\sqrt7)^2}{4\sqrt{35}}
\quad\text{et}\quad
B=\frac{4}{\sqrt2-\sqrt3}+4\sqrt2+4\sqrt3
\]
sont des entiers naturels non nuls. 

\section*{Exercice 21}
Soit $\rho=\dfrac{1+\sqrt7}{2}$.
\begin{enumerate}
  \item Montrer $2\rho^2=2\rho+3$.
  \item En déduire $2\rho^3=5\rho+3$.
\end{enumerate}

\section*{Exercice 22}
Soient $x=12^6$, $y=6^8$, $z=2^{11}\times3^7$. Vérifier $x^x y^y=z^z$.

\section*{Exercice 23}
Soient $a,b\in\mathbb{R}$.
\begin{enumerate}
  \item Montrer $\max(a,b)+\min(a,b)=a+b$.
  \item Montrer $\max(a,b)-\min(a,b)=|a-b|$.
  \item En déduire les formules explicites de $\min$ et $\max$.
\end{enumerate}

\end{document}
