\documentclass[12pt,a4paper]{article}
\usepackage[utf8]{inputenc}
\usepackage[T1]{fontenc}
\usepackage[french]{babel}
\usepackage{amsmath,amssymb}
\usepackage{lmodern}
\usepackage{geometry}
\geometry{margin=2.5cm}

\title{Logique}
\author{}
\date{}

\begin{document}

\maketitle

\section*{Exercice 1}
Soit $a$ un nombre entier multiple de 7 et $b$ un nombre entier qui n’est pas un multiple de 7.  
Démontrer que $a+b$ n’est pas un multiple de 7.

\section*{Exercice 2 (Propositions)}
Parmi les phrases ci-dessous, quelles sont celles qui sont des propositions mathématiques ?
\begin{enumerate}
\item « 42 est le triple de 14 »
\item « Le périmètre du rectangle $ABCD$ »
\item « La moitié de 17 n’est pas égale à 8 »
\item « 7/42 »
\item « Le point $P$ est perpendiculaire à $(KL)$ »
\item « $6 \times 7 = 52$ »
\item « $2~m \times 9~m = 18~m^2$ »
\item « Les diagonales du rectangle $ABC$ ont la même longueur »
\item « 4,16 $<$ 4,106 »
\item « Un carré est un rectangle particulier »
\item « $1/0$ est positif »
\item « Le presbytère n’a rien perdu de son charme ni le jardin de son éclat »
\item « $8x+5=4-x$ »
\item « Il existe un nombre $x$ tel que $x^2<0$ »
\item « Quels que soient les nombres $a$ et $b$, $(a+b)^2=a^2+2ab+b^2$ »
\end{enumerate}

\section*{Exercice 3 (Négation d’une proposition)}
\begin{enumerate}
\item Écrire une négation pour chacune des propositions suivantes :
  \begin{enumerate}
  \item « Toutes les voitures rapides sont rouges. »
  \item « Tout triangle rectangle possède un angle droit. »
  \item « Il existe un mouton écossais dont au moins un côté est noir. »
  \item « Dans toutes les prisons tous les détenus détestent tous les gardiens. »
  \end{enumerate}

\item $a$ désigne un nombre réel. Écrire une négation pour chacune des propositions suivantes :
  \begin{enumerate}
  \item $a>5$
  \item $a \geq -4$
  \item $a \leq -2$ ou $a > 3$
  \item $a < 5$ et $a > -1$
  \end{enumerate}

\item Écrire une négation pour chacune des propositions suivantes :
  \begin{enumerate}
  \item « Pour tout nombre $x$, $x^2>0$ »
  \item « Il existe un nombre $x$ tel que $x^2<x$ »
  \item « Pour tous nombres $a$ et $b$, $(a+b)^2=a^2+b^2$ »
  \item « Il existe $a$ et $b$ tels que $a^2+b^2=13$ »
  \item « La Terre tourne autour du Soleil »
  \end{enumerate}
\end{enumerate}

\section*{Exercice 4 (Vrai ou Faux)}
Dire si les affirmations suivantes sont vraies ou fausses.
\begin{enumerate}
\item Le drapeau de l’Autriche est blanc et rouge.
\item L’eau bout à 100 degrés $\implies$ la Terre tourne autour du Soleil.
\end{enumerate}

\section*{Exercice 6}
Compléter le tableau suivant par vrai ou faux :

\begin{center}
\begin{tabular}{|c|l|l|c|c|c|}
\hline
 & A & B & $A \Rightarrow B$ & $B \Rightarrow A$ & $A \Leftrightarrow B$ \\
\hline
1 & Je réside en France & Je réside en Europe & & & \\
2 & Je ne réside pas en Europe & Je ne réside pas en France & & & \\
3 & Je suis majeur & J’ai 19 ans & & & \\
4 & $CDEF$ est un parallélogramme & $CDEF$ est un carré & & & \\
5 & $x=3$ & $x^2=9$ & & & \\
6 & $MNP$ est rectangle en $M$ & $MP^2+MN^2=NP^2$ & & & \\
7 & $x \geq -2$ & $x \geq -1$ & & & \\
8 & $x \geq -2$ & $x > -2$ & & & \\
9 & $a+b=5$ & $a=2$ et $b=3$ & & & \\
10 & $4x-(x-5)=7$ & $x=\tfrac{2}{3}$ & & & \\
11 & $n$ est premier & $n$ n’est pas un multiple de 3$ $ & & & \\
12 & $n$ est pair & $\tfrac{1}{n}$ est décimal & & & \\
13 & $(ax+b)(cx+d)=0$ & $ax+b=0$ ou $cx+d=0$ & & & \\
\hline
\end{tabular}
\end{center}

\section*{Exercice 7}
Indiquer — en justifiant la réponse — si chaque proposition est vraie ou fausse.
\begin{enumerate}
\item Tous les multiples de 3 sont des multiples de 9.
\item Tous les diviseurs de 12 sont des diviseurs de 36.
\item Le carré de la somme de deux nombres est égal à la somme des carrés des deux nombres.
\item Un carré est un rectangle.
\item Pour tout réel $x$ tel que $x^2>4$ alors $x>2$.
\item Il existe une puissance de 2 qui s’écrit avec un 7 comme chiffre de gauche.
\item Il existe une puissance de 7 qui admet 2017 comme derniers chiffres de droite.
\item Il existe une puissance de 113 qui admet 2017 comme derniers chiffres de droite.
\item $(4+\sqrt{17})^8=18957314$.
\end{enumerate}

\section*{Exercice 8}
\begin{enumerate}
\item Pour chacune des propositions ci-dessous, dire si cette proposition est vraie ou fausse.
\item Énoncer la proposition réciproque et dire si elle est vraie ou fausse.
\item Dire dans quel cas on a une équivalence.
\end{enumerate}

\begin{enumerate}
\item[(i)] Si je suis autrichien, alors je suis européen.
\item[(ii)] Si $x^2=4$ alors $x=2$.
\item[(iii)] Si $ab=0$ alors $a=0$ ou $b=0$.
\item[(iv)] Si $ABCD$ est un losange alors $ABCD$ est un parallélogramme.
\item[(v)] Si deux droites sont perpendiculaires alors elles sont sécantes.
\item[(vi)] Si $ABC$ est un triangle équilatéral alors $ABC$ est un triangle isocèle.
\end{enumerate}

\section*{Exercice 9 (Jeux olympiques)}
Lors de la cérémonie d’ouverture des jeux olympiques, les athlètes japonais portent tous une chemise rouge.
\begin{enumerate}
\item Un athlète qui porte une chemise blanche entre dans le stade. Est-il un athlète japonais ?
\item À côté de l’athlète précédent, on voit quelqu’un qui porte une chemise rouge. Est-il un athlète japonais ?
\item Le haut-parleur annonce l’arrivée d’un athlète chinois. Porte-t-il une chemise rouge ?
\item Sur la pelouse du stade on voit un athlète japonais qui porte un manteau. Porte-t-il une chemise rouge ?
\end{enumerate}

\section*{Exercice 10}
\begin{enumerate}
\item Pour chacune des propositions $(P_1)$ ci-dessous, énoncer sa contraposée $(P_2)$.
\item Pour chacune des propositions $(P_1)$ ci-dessous, énoncer sa réciproque $(P_3)$.
\item Pour chacune des propositions $(P_1)$ ci-dessous, énoncer la contraposée de la réciproque $(P_4)$.
\item Parmi les différentes propositions $(P_1)$, $(P_2)$, $(P_3)$ et $(P_4)$, dire lesquelles sont vraies.
\end{enumerate}

\begin{enumerate}
\item[(a)] Si un quadrilatère est un rectangle alors ses diagonales sont égales.
\item[(b)] Si un nombre se termine par 5 alors il est divisible par 5.
\item[(c)] Si un triangle est rectangle alors il possède deux angles égaux.
\item[(d)] Si deux droites sont perpendiculaires alors elles sont sécantes.
\end{enumerate}

\end{document}
