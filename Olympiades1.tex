\documentclass[a4paper,11pt]{article}
\usepackage[utf8]{inputenc}
\usepackage[T1]{fontenc}
\usepackage[french]{babel}
\usepackage{amsmath,amssymb}
\usepackage{forest}

\begin{document}

\section*{Génération à la suite}

\subsection*{Les fils d’une fraction strictement positive}

Soient $i$ et $j$ des entiers strictement positifs.  
On dit que les deux fils de la fraction $\dfrac{i}{j}$ sont les fractions 
\[
\dfrac{i}{i+j} \quad \text{et} \quad \dfrac{i+j}{j}.
\]

On dit que $\dfrac{i}{i+j}$ est le \textbf{fils gauche} et que $\dfrac{i+j}{j}$ est le \textbf{fils droit} de $\dfrac{i}{j}$.

L’arbre ci-dessous illustre la situation :

\[
\begin{forest}
for tree={
  align=center,
  math content,
  l sep=8mm, s sep=10mm,
  edge={-latex}, % petites flèches
}
[\dfrac{i}{j}
  [\dfrac{i}{i+j}]
  [\dfrac{i+j}{j}]
]
\end{forest}
\]

Par exemple, $\dfrac{7}{5}$ a pour fils gauche
\[
\dfrac{7}{7+5} = \dfrac{7}{12},
\]
et pour fils droit
\[
\dfrac{7+5}{5} = \dfrac{12}{5}.
\]

\bigskip

\textbf{Q1 :} Quels sont les deux fils de $\dfrac{8}{9}$ ?  

\textbf{Q2 :} Trouver la fraction dont un des fils est $\dfrac{3}{8}$.  

\textbf{Q3 :} Trouver la fraction dont un des fils est $\dfrac{111}{7}$.  

\textbf{Q4 :} De manière générale, justifier que l’un des fils de $\dfrac{i}{j}$ est inférieur strictement à $1$ et que l’autre est supérieur strictement à $1$.

\subsection*{Arbre et suite de Calkin-Wilf}

L'\textbf{arbre de Calkin-Wilf} s’obtient en prenant la fraction 
\[
\dfrac{1}{1}
\]
à la racine et en associant à chaque fraction ses deux fils (comme dans la partie A).

\begin{center}
\begin{forest}
for tree={
  align=center,
  math content,
  l sep=10mm, s sep=12mm,
  rectangle, draw, rounded corners, inner sep=2pt
}
[\dfrac{1}{1}
  [\dfrac{1}{2}
    [\dfrac{1}{3}
      [,draw,rectangle,minimum width=8mm,minimum height=6mm]
      [,draw,rectangle,minimum width=8mm,minimum height=6mm]
    ]
    [\dfrac{3}{2}
      [,draw,rectangle,minimum width=8mm,minimum height=6mm]
      [,draw,rectangle,minimum width=8mm,minimum height=6mm]
    ]
  ]
  [\dfrac{2}{1}
    [\dfrac{2}{3}
      [,draw,rectangle,minimum width=8mm,minimum height=6mm]
      [,draw,rectangle,minimum width=8mm,minimum height=6mm]
    ]
    [\dfrac{3}{1}
      [,draw,rectangle,minimum width=8mm,minimum height=6mm]
      [,draw,rectangle,minimum width=8mm,minimum height=6mm]
    ]
  ]
]
\end{forest}
\end{center}
\end{itemize}


\medskip

\textbf{Q5 :} Donner, sur sa copie, toutes les fractions de la ligne 4 de cet arbre.  

\textbf{Q6 :} Montrer que la fraction $\dfrac{44}{13}$ apparaît dans l’arbre. Sur quelle ligne apparaît-elle ?  

\medskip

La \textbf{suite de Calkin-Wilf} est la suite $(u_n)_{n \geq 1}$ des fractions lues de gauche à droite, ligne par ligne en descendant l’arbre de Calkin-Wilf.  

Ainsi, les premiers termes de la suite $(u_n)$ sont :  
\[
u_1 = \dfrac{1}{1}, \quad
u_2 = \dfrac{1}{2}, \quad
u_3 = \dfrac{2}{1}, \quad
u_4 = \dfrac{1}{3}, \quad
u_5 = \dfrac{3}{2}, \quad
u_6 = \dfrac{2}{3}, \quad
u_7 = \dfrac{3}{1}.
\]

\textbf{Q7 :} Donner la fraction égale à $u_{32}$.  

\textbf{Q8 :} On constate que les deux fils de $u_3$ sont $u_6$ et $u_7$.  

Recopier et compléter, sur sa copie, les deux phrases suivantes (aucune justification n’est attendue) :  

\begin{itemize}
    \item « Les deux fils de $u_6$ sont $u_{...}$ et $u_{...}$. »  
    \item « Si $n$ désigne un entier naturel supérieur ou égal à 1, les deux fils de $u_n$ sont $u_{...}$ et $u_{...}$. »  
\end{itemize}

\subsection*{Suite de Stern}

La \textbf{suite de Stern} est la suite $(v_n)_{n \geq 1}$ dont les termes sont les numérateurs des fractions lues de gauche à droite, ligne par ligne en descendant l’arbre de Calkin-Wilf.  

On obtient ainsi :  
\[
v_1 = 1, \quad v_2 = 1, \quad v_3 = 2, \quad v_4 = 1, \quad v_5 = 3, \quad v_6 = 2, \quad v_7 = 3.
\]

\textbf{Q9 :} Donner les valeurs numériques de $v_8$ à $v_{15}$.  

On admet que pour tout entier naturel $n$ non nul, le dénominateur de la $n$-ième fraction de la suite de Calkin-Wilf est aussi le numérateur de la $(n+1)$-ième.  

Autrement dit, on admettra que pour tout entier naturel $n$ non nul, on a :  
\[
u_n = \dfrac{v_n}{v_{n+1}}.
\]

\textbf{Q10 :} Utiliser les résultats précédents pour montrer que, pour tout entier $n$ supérieur ou égal à $1$, on a les égalités suivantes :  
\[
v_{2n} = v_n, 
\qquad 
v_{2n+1} = v_n + v_{n+1}.
\]

\textbf{Q11 :} En déduire les valeurs numériques de $v_{64}$ et $v_{65}$.

\end{document}
