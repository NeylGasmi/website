\documentclass[11pt,a4paper]{article}

\usepackage[T1]{fontenc}
\usepackage[utf8]{inputenc}
\usepackage[french]{babel}
\usepackage{amsmath,amssymb}
\usepackage{enumitem}
\usepackage[margin=2.5cm]{geometry}
\usepackage{xcolor}

\setlength{\parindent}{0pt}
\setlength{\parskip}{6pt}
\setlist[enumerate]{itemsep=6pt, topsep=4pt}

\newcommand{\hl}[1]{\colorbox{blue!15}{#1}}

\begin{document}

\section*{Énoncé}

On plie une feuille de papier rectangulaire le long d'une de ses diagonales ; on coupe les parties qui ne se recouvrent pas puis on déplie la feuille.

\textit{On admet qu'ainsi on obtient toujours un \hl{losange}} (cette propriété sera démontrée dans la dernière question de l'exercice).

L'unité de longueur choisie est le centimètre.

\begin{enumerate}
  \item Construire le losange obtenu à partir d'une feuille rectangulaire de longueur $L=16$ et de largeur $\ell=8$.\\
  On pourra noter $c$ la longueur du côté du losange.

  \item \textit{Les questions suivantes sont indépendantes.}

  \item Dans cette question, la feuille rectangulaire de départ a pour longueur $16$ et pour largeur $8$. Calculer la longueur du côté du losange.

  \item On veut maintenant obtenir un losange de côté $7{,}5$ à partir d'une feuille dont les dimensions (longueur et largeur) sont des nombres entiers. Quelles sont les dimensions possibles pour la feuille de départ ?

  \item À partir d'une feuille de longueur $L$, on a obtenu un losange dont l'aire est égale à $75\,\%$ de celle de la feuille de départ. Exprimer, en fonction de $L$, la largeur $\ell$ de la feuille de départ.

  \item Démontrer le résultat admis initialement, à savoir que la manipulation décrite en début d'énoncé conduit toujours à un losange.
\end{enumerate}

\end{document}
