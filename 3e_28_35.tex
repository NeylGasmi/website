\documentclass[a4paper,11pt]{article}
\usepackage[utf8]{inputenc}
\usepackage[T1]{fontenc}
\usepackage[french]{babel}
\usepackage{amsmath,amssymb}
\usepackage{geometry}
\geometry{margin=2cm}
\setlength{\parindent}{0pt}

\begin{document}
Neyl Gasmi
\section*{Exercices 28 à 35}

\subsection*{Ex 28}
Un entier naturel non nul est écrit sur chacune des faces d’un cube, et sur chaque sommet on écrit le produit des nombres inscrits sur les trois faces adjacentes à ce sommet.  

La somme des nombres placés aux sommets du cube est 105.  
Quelle est la somme des nombres placés sur les faces du cube ?

\bigskip

\subsection*{Ex 29 \; En facteurs premiers}
Décomposer en produit de facteurs premiers :  
\[
A = 24\,999\,999, 
\qquad 
B = 1\,018\,081
\]

\bigskip

\subsection*{Ex 30}
Résoudre les équations.
\begin{enumerate}
  \item \(1 - 2x + 3 - 5x = -x - 1 + 2 - 4x\)
  \item \(-5x + 1 - x + 3 - 4x + 1 = 0\)
  \item \((2x+1) - 3(5x+1) = 2(x-4) - (3x-6)\)
  \item \(3x - 4(x+2) = x + 3 - (7 - 6x)\)
  \item \(7 - (2x-3) + x = x - 1 - 3(2x+1)\)
  \item \(4 - (-2x - (5+4x)) = 5x - (3 - 2(4x-1))\)
\end{enumerate}

\bigskip

\subsection*{Ex 31}
Résoudre les équations.
\begin{enumerate}
  \item \(\dfrac{x-3}{4} = x+3\)
  \item \(\dfrac{1}{2}x+2 = \dfrac{x-1}{3}\)
  \item \(\dfrac{2x-1}{3} = \dfrac{-5-x}{4}\)
  \item \(\dfrac{2x-3}{4} = \dfrac{3x-1}{2}\)
  \item \(\dfrac{2}{3}x - \dfrac{1}{4} = \dfrac{1}{2} + \dfrac{x}{6}\)
  \item \(\dfrac{3}{8}x - \dfrac{1}{2} = \dfrac{1}{2}x - \dfrac{2}{3}\)
  \item \(\dfrac{x}{2} - 1 = \dfrac{7x-4}{8}\)
  \item \(\dfrac{5}{6}x - \dfrac{1}{3} = \dfrac{2}{3}x - \dfrac{1}{2}\)
\end{enumerate}

\bigskip

\subsection*{Ex 32 \; Problèmes géométriques}
\begin{enumerate}
  \item L’aire d’un trapèze est de \(85{,}5\ \text{cm}^2\). Sa hauteur est de \(4{,}5\ \text{cm}\). Une de ses bases mesure \(15\ \text{cm}\). Calculer la longueur de l’autre base.
  \item Le périmètre d’un rectangle mesure \(240\ \text{m}\). Sa longueur mesure \(26\ \text{m}\) de plus que sa largeur. Calculer ses dimensions.
  \item Combien mesure le côté d’un triangle équilatéral dont une hauteur mesure \(6\ \text{cm}\) ?
  \item Un rectangle a \(15\ \text{m}\) de largeur. Si on diminuait sa longueur de \(14\ \text{m}\) et si on augmentait sa largeur de \(6\ \text{m}\), l’aire ne varierait pas. Calculer la longueur de ce rectangle.
  \item Dans un losange, la grande diagonale mesure \(7\ \text{cm}\) de plus que la petite. Si on diminuait la longueur de la grande diagonale de \(9\ \text{cm}\) et si on augmentait la longueur de la petite diagonale de \(5\ \text{cm}\), l’aire diminuerait de \(82\ \text{cm}^2\). Calculer la longueur de chaque diagonale.
\end{enumerate}

\bigskip

\subsection*{Ex 33 \; Problèmes d’âge}
\begin{enumerate}
  \item L’âge d’un père est le quadruple de celui de son fils. Quel est l’âge du père, sachant que, dans 20 ans, il ne sera plus que le double de celui de son fils ?
  \item Bob a le double de l’âge de Joe. Il y a 10 ans, Bob avait quatre fois l’âge de Joe. Quels sont les âges de Bob et de Joe ?
  \item Il y a 55 ans, l’âge d’un père dépassait de 25 ans l’âge de son fils. Dans 14 ans, l’âge du fils sera égal aux trois quarts de l’âge de son père. Quels sont les âges du père et du fils ?
\end{enumerate}

\bigskip

\subsection*{Ex 34 \; Brevet 2003}
Soit l’expression :  
\[
A = 9x^2 - 49 + (3x+7)(2x+3).
\]
\begin{enumerate}
  \item Développer l’expression \(A\).
  \item Factoriser \(9x^2 - 49\), puis l’expression \(A\).
  \item Résoudre l’équation \(A = 0\).
\end{enumerate}

\bigskip

\subsection*{Ex 35 \; Brevet 1998}
On considère l’expression :  
\[
E = (3x+2)^2 - (x-1)^2.
\]
\begin{enumerate}
  \item Développer et réduire \(E\).
  \item Factoriser \(E\).
  \item Résoudre l’équation \(E = 0\).
\end{enumerate}

\end{document}
