\documentclass[11pt,a4paper]{article}
\usepackage[french]{babel}
\usepackage[T1]{fontenc}
\usepackage[utf8]{inputenc}
\usepackage{lmodern}
\usepackage{amsmath,amssymb}
\usepackage{microtype}
\usepackage{geometry}
\geometry{margin=2.2cm}
\usepackage{enumitem}
\setlist[enumerate,1]{label=\textbf{\arabic*.}, leftmargin=0pt, itemindent=2.2em, labelsep=.6em}
\setlist[enumerate,2]{label=\textbf{\alph*)}, leftmargin=*, itemsep=.2em}
\usepackage[most]{tcolorbox}
\tcbset{boxrule=.4pt,arc=2pt,enhanced}

\newcommand{\trip}[3]{\((#1\,;\,#2\,;\,#3)\)}
\newcommand{\Ep}{E_p}

\begin{document}

\begin{tcolorbox}[colback=black!2!white,colframe=black!50!white]
\large\textbf{Exercice 1 — Triangles à côtés entiers}
\end{tcolorbox}

On dit qu’un triangle est \emph{entier} si les longueurs de ses trois côtés sont des entiers
naturels non nuls. On rappelle la propriété dite de l’\textbf{inégalité triangulaire} : dans tout triangle non aplati,
la longueur de chaque côté est strictement inférieure à la somme des longueurs des deux autres.

\begin{enumerate}
\item
\begin{enumerate}
  \item Parmi les triplets suivants \((x;y;z)\), expliquer lequel désigne les longueurs
  des côtés d’un triangle entier non aplati, puis comment tracer ce triangle et avec quels outils :
  \trip{4}{4}{5}, \trip{3}{6}{9}, \trip{2}{2}{6}.
  \item Quelles sont les valeurs possibles de l’entier \(z\) si \trip{15}{19}{z} désigne les longueurs
  des trois côtés d’un triangle entier non aplati rangées par ordre croissant ?
  \item Étant donnés trois entiers naturels non nuls \(x,y,z\) tels que \(x\le y\le z\),
  quelle condition faut-il ajouter pour que le triplet \trip{x}{y}{z} désigne
  les longueurs des côtés d’un triangle entier non aplati ?
\end{enumerate}

\item Soit \(p\) un entier naturel non nul. On désigne par \(\Ep\) l’ensemble des triplets d’entiers
naturels rangés par ordre croissant \(x\le y\le z\) et désignant les côtés d’un triangle entier non aplati
de \textbf{périmètre} égal à \(p\). Ainsi on obtient, par exemple,
\[
E_{9}=\{\trip{1}{4}{4},\,\trip{2}{3}{4},\,\trip{3}{3}{3}\}.
\]
\begin{enumerate}
  \item Si un triplet appartient à \(E_{18}\), quelles sont les valeurs maximale et minimale possibles pour \(z\) ?
  \item Donner la composition de \(E_{18}\) et représenter, dans un repère orthonormé,
  l’ensemble des couples \((x,y)\) pour lesquels il existe un entier naturel \(z\) tel que
  \trip{x}{y}{z}\(\in E_{18}\).
  Vérifier que ces couples se situent à l’intérieur ou sur les bords d’un triangle dont les
  sommets ont des coordonnées entières.
  \end{enumerate}
  
 \item \textbf{}
\begin{enumerate}
  \item Justifier que si \trip{x}{y}{z}\(\in E_p\) alors \(\trip{x+1}{y+1}{z+1}\in E_{p+3}\).
  \item Soit \trip{x}{y}{z}\(\in E_{p+3}\). Déterminer une condition sur \((x,y,z)\) pour que
  \(\trip{x-1}{y-1}{z-1}\in E_{p}\).
  \item En déduire que si \(p\) est impair, alors \(E_p\) et \(E_{p+3}\) ont le même nombre d’éléments.
\end{enumerate}

\item \textbf{Étude de \(E_{2019}\).}
\begin{enumerate}
  \item \(E_{2019}\) contient-il un triplet \trip{x}{y}{z} correspondant à un triangle équilatéral ?
  \item \(E_{2019}\) contient-il des triplets correspondant à des triangles isocèles non équilatéraux ?
  Si oui, combien ?
  \item Montrer que si \(E_{2019}\) contient un triplet \trip{x}{y}{z} correspondant à un triangle rectangle,
  alors
  \[
  2019^2=4038(x+y)-2xy.
  \]
  \item En déduire que \(E_{2019}\) ne contient pas de triangle rectangle.
\end{enumerate}

\item \textbf{Dénombrement de \(E_{2022}\).}
\begin{enumerate}
  \item Soit \trip{x}{y}{z}\(\in E_{2022}\). On rappelle \(x\le y\le z\).
  Justifier que \(x+y\ge1012\) et \(x+2y\le2022\).
  \item Réciproquement, montrer que si \(x\le y\), \(x+y\ge1012\) et \(x+2y\le2022\),
  alors \trip{x}{y}{2022-x-y}\(\in E_{2022}\).
  \item Justifier que, dans un repère orthonormé, l’ensemble des couples d’entiers naturels \((x,y)\)
  tels que \(x\le y\), \(x+y\ge1012\) et \(x+2y\le2022\) constitue l’ensemble des points
  à coordonnées entières d’un triangle.
  Évaluer son aire et le nombre de points à coordonnées entières situés sur ses côtés.
  \end{enumerate}

 \item  On admet le Théorème de Pick:
  \begin{tcolorbox}[colback=green!6!white,colframe=green!40!black,title=Théorème de Pick]
    Si un polygone \(P\) est tel que tous ses sommets sont à coordonnées entières
    dans un repère orthonormé, alors son aire est donnée par
    \[
      A = i + \frac{j}{2} - 1,
    \]
    où \(i\) désigne le nombre de points à coordonnées entières situés à l’intérieur de \(P\)
    et \(j\) le nombre de ceux situés sur les côtés de \(P\).
  \end{tcolorbox}
En déduire le nombre de triplets de \(E_{2022}\), puis celui de \(E_{2019}\).
\end{enumerate}

\item \textbf{Une solution algorithmique.}
\begin{enumerate}
  \item De manière générale, concevoir un programme (à retranscrire sur la copie) qui énumère et dénombre les éléments de \(\Ep\).
Le tester sur \(E_{2022}\) puis sur \(E_{2019}\).
\end{enumerate}

\end{enumerate}

\end{document}
