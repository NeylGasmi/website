\documentclass[a4paper,11pt]{article}
\usepackage[utf8]{inputenc}
\usepackage[T1]{fontenc}
\usepackage[french]{babel}
\usepackage{amsmath,amssymb}
\usepackage{siunitx} % pour \mu s et les puissances de 10 propres
\usepackage{geometry}
\geometry{margin=2cm}
\setlength{\parindent}{0pt}

\begin{document}

{\Large \textbf{Exercices — Calcul numérique et fractions}}\par\medskip

% ===========================
\section*{Exercice 1}
\textbf{Calculer chaque expression algébrique.}

\[
\begin{aligned}
\text{A }&= -3-5
&\qquad \text{B }&= -3\times(-5)
&\qquad \text{C }&= -3-(-5)
&\qquad \text{D }&= -8+2\\[2pt]
\text{E }&= 8:(-2)
&\qquad \text{F }&= -10-20
&\qquad \text{G }&= -10\times 20
&\qquad \text{H }&= -5-6
\end{aligned}
\]

% ===========================
\section*{Exercice 2}
\textbf{Retirer les parenthèses puis calculer.}

\[
\text{A }= 36-26+17-33
\qquad\qquad
\text{B }= -17-9-13-(-15)+14
\]

% ===========================
\section*{Exercice 3}
\textbf{Sans effectuer de calculs, déterminer le signe de l'expression.}

\[
\begin{aligned}
\text{A }&= (-5)\times(-6)\times 7
&\qquad
\text{B }&= 3\times(-2)\times 5\times(-1)\\[2pt]
\text{C }&= (-25:5)\times\bigl[-7:(-2)\bigr]
&\qquad
\text{D }&= -1\times\bigl(5:(-3)\bigr)
\end{aligned}
\]

% ===========================
\section*{Exercice 4}
\textbf{Bien détailler les calculs en respectant les priorités opératoires.}

\[
\begin{aligned}
\text{A }&= 2\times(-3)-3\times(-7)
&\qquad
\text{B }&= -3-5\times(-2)\\[2pt]
\text{C }&= 6\times 5 - 7\times 9 + 4\times(-3)
&\qquad
\text{D }&= 4\times(-6-8\times 2)-10
\end{aligned}
\]

\bigskip
{\Large \textbf{Exercices — Fractions}}\par\medskip

% ===========================
\section*{Exercice 5}
\textbf{Compléter le tableau en détaillant bien les calculs.}

\medskip
\renewcommand{\arraystretch}{1.4}
\begin{tabular}{|c|c|c|c|c|}
\hline
 &  &  &  &  \\
\(\mathbf{a}\) & \(\displaystyle -\frac{3}{4}\) & \(\displaystyle -\frac{8}{15}\) & \(\displaystyle \frac{2}{7}\) & \(\displaystyle \frac{5}{6}\)\\
\hline
\(\mathbf{b}\) & \(\displaystyle \frac{7}{4}\) & \(\displaystyle -\frac{2}{3}\) & \(\displaystyle \frac{5}{9}\) & \(\displaystyle -\frac{3}{4}\)\\
\hline
\(\mathbf{a+b}\) &  &  &  &  \\
\hline
\(\mathbf{a-b}\) &  &  &  &  \\
\hline
\end{tabular}

% ===========================
\section*{Exercice 6}
\textbf{Calculer chaque produit et donner le résultat sous forme irréductible.}

\[
\begin{aligned}
\text{A }&= \left(-\frac{3}{5}\right)\times\frac{7}{12}
&\qquad
\text{B }&= \left(\frac{-3}{-7}\right)\times\left(\frac{-8}{15}\right)
&\qquad
\text{C }&= \left(\frac{5}{-6}\right)\times 18
&\qquad
\text{D }&= \left(\frac{-15}{8}\right)\times\left(\frac{27}{-12}\right)\times\left(\frac{-7}{2}\right)
\end{aligned}
\]

% ===========================
\section*{Exercice 7}
\textbf{Calculer chaque quotient et donner le résultat sous forme irréductible.}

\[
\begin{aligned}
\text{A }&= \dfrac{5}{7}\div\dfrac{15}{8}
&\qquad
\text{B }&= \dfrac{24}{6}\div\left(\dfrac{-9}{11}\right)
&\qquad
\text{C }&= \dfrac{-11}{-18}\div\dfrac{-8}{15}
&\qquad
\text{D }&= \dfrac{-7}{6}\div\dfrac{-4}{15}
\end{aligned}
\]

% ===========================
\section*{Exercice 8}
\textbf{Calculer chaque expression en détaillant bien les étapes de calcul.}

\[
\begin{aligned}
\text{A }&= \dfrac{8}{3}-\dfrac{8}{3}\times\dfrac{9}{16}
&\qquad
\text{B }&= \left(\dfrac{3}{4}-\dfrac{11}{8}\right)\div\left(\dfrac{5}{3}-\dfrac{7}{4}\right)
&\qquad
\text{C }&= \left(\dfrac{8}{7}-\dfrac{6}{5}\right)\times\dfrac{7}{4}-2
\end{aligned}
\]

\bigskip
{\Large \textbf{Exercices — Puissances de 10 et grandeurs}}\par\medskip

% ===========================
\section*{Exercice 9}
Donner l’écriture décimale des nombres suivants :
\[
\text{a) }10^{4} \qquad
\text{b) }10^{-2} \qquad
\text{c) }10^{6} \qquad
\text{d) }10^{-1}.
\]

% ===========================
\section*{Exercice 10}
Le sprinter Usain Bolt parcourt \(1\,\text{m}\) en \(9{,}6\times 10^{-2}\,\text{s}\).\\
La fusée Apollo~10 parcourt \(1\,\text{m}\) en \(90\,\mu\text{s}\).\\
Lino affirme : « La fusée Apollo~10 va \(1000\) fois plus vite qu'Usain Bolt. »\\
A-t-il raison ? Justifier.

% ===========================
\section*{Exercice 11}
L’unité de production électrique est le wattheure (Wh).\\
Exprimer chacune des productions suivantes en wattheures (Wh). En 2019, en France :
\begin{itemize}
  \item le nucléaire a produit \(416\,\text{TWh}\) ;
  \item l’hydraulique a produit \(68\,200\,\text{GWh}\) ;
  \item l’éolien a produit \(17\,000\,000\,\text{MWh}\).
\end{itemize}

\section*{Exercice 12}
Donner les écritures décimales des produits suivants :
\[
\begin{aligned}
\text{a) }& 452\times 10^{-2}
&\qquad \text{b) }& 31{,}5\times 10^{4}\\[4pt]
\text{c) }& 0{,}0067\times 10^{-1}
&\qquad \text{d) }& 0{,}902\times 10^{8}
\end{aligned}
\]

% ===========================
\section*{Exercice 13}
Quelle est l’écriture scientifique des nombres suivants ?
\[
A=34{,}7 
\qquad B=0{,}0845
\qquad C=46{,}121\times 10^{3}
\qquad D=0{,}078\times 10^{-3}
\]

\bigskip
{\Large \textbf{Exercices — Développement et factorisation}}\par\medskip

% ===========================
\section*{Exercice 14}
Développer et réduire.
\[
\text{a) } 5(x-3)
\qquad \text{b) } -2(3x+5)
\qquad \text{c) } 3x(-2x+1)
\]

% ===========================
\section*{Exercice 15}
Factoriser.
\[
\text{a) } 3x-21
\qquad \text{b) } x^{2}-2x
\qquad \text{c) } 5+5x
\]

% ===========================
\section*{Exercice 3}
Développer et réduire les expressions suivantes :
\[
A=3x-4(x-5)
\qquad
B=-2x(3x-1)+8(1+2x)
\]

\bigskip
{\Large \textbf{Exercices — Équations}}\par\medskip

% ===========================
\section*{Exercice 16}
Dans chaque cas, dire si la valeur donnée est solution de l’équation.
\[
\text{a) } -5 \quad \text{est-il solution de } 2x-6=-9\;?
\qquad
\text{b) } 0{,}5 \quad \text{est-il solution de } 3x+1=-5x+5\;?
\]

% ===========================
\section*{Exercice 17}
Résoudre chacune des équations suivantes en détaillant bien les étapes :
\[
\begin{aligned}
\text{a) }& 3x=8 
&\qquad \text{b) }& x-4=-1
&\qquad \text{c) }& 3x+2=5\\[6pt]
\text{d) }& x-2=6x+3 
&\qquad \text{e) }& 4x-7=-3x+1
\end{aligned}
\]

\bigskip
{\Large \textbf{Exercices — Pourcentages}}\par\medskip

% ===========================
\section*{Exercice 18}
Calculer :
\[
\text{a) }45\%\ \text{de } 80\ \text{élèves}
\qquad
\text{b) }60\%\ \text{de }70\,€
\qquad
\text{c) }15\%\ \text{de }3600\ \text{animaux}
\]

% ===========================
\section*{Exercice 19}
On compte environ \(25\,823\,000\) actifs en France.
\begin{enumerate}
  \item Sachant qu’il y a \(2,8\%\) d’agriculteurs, calculer combien cela représente de personnes.
  \item Le nombre de personnes travaillant dans la construction est de \(1\,704\,300\).  
  Calculer leur pourcentage à \(0,1\%\) près par rapport au nombre d’actifs.
\end{enumerate}

% ===========================
\section*{Exercice 20}
\begin{enumerate}
  \item Lors d’une élection dans une commune où \(480\) votes ont été exprimés, une candidate a obtenu \(11,25\%\) des voix. Calculer le nombre de personnes qui ont voté pour elle.
  \item Pour la même élection, un autre candidat a obtenu \(132\) voix. Calculer le pourcentage de voix obtenus par ce candidat.
\end{enumerate}

\bigskip
{\Large \textbf{Exercices — Triangles et Pythagore}}\par\medskip

% ===========================
\section*{Exercice 21}
\(ARC\) est un triangle rectangle en \(R\) tel que \(AC=52\ \text{mm}\) et \(RC=48\ \text{mm}\).  
Calculer la longueur \(AR\).

% ===========================
\section*{Exercice 22}
Le triangle \(PIE\) est rectangle en \(I\) tel que : \(IP=7\ \text{cm}\) et \(IE=4\ \text{cm}\).  
Quelle est la valeur exacte de \(PE\) ?

% ===========================
\section*{Exercice 23}
Soit \(MNP\) un triangle tel que \(MN=9{,}6\ \text{cm}\), \(MP=4\ \text{cm}\) et \(NP=10{,}3\ \text{cm}\).  
Ce triangle est-il rectangle ?

{\Large \textbf{Exercices — Volumes, proportionnalité, statistiques}}\par\medskip

% ===========================
% VOLUMES (pots, pyramide, verre)
% ===========================
\section*{Exercice 24}
Léo a obtenu \SI{2,7}{\liter} de confiture.\\
Il la verse dans des pots cylindriques de \SI{6}{\centi\meter} de diamètre et de \SI{12}{\centi\meter} de haut, qu’il remplit jusqu’à \SI{1}{\centi\meter} du bord.
\begin{enumerate}
  \item Combien pourra-t-il remplir de pots ?
  \item Il colle ensuite sur ses pots une étiquette rectangulaire de fond blanc qui recouvre toute la surface latérale du pot.\\
  Montrer que la longueur de l’étiquette est d’environ \SI{18,8}{\centi\meter}.
\end{enumerate}

\bigskip
\section*{Exercice 25}
Voici les subventions du conseil général pour deux collèges :\\[2pt]
\quad Collège \textbf{A. Daudet} : \SI{1\,430\,000}{\euro} pour \(\,650\) élèves.\\
\quad Collège \textbf{V. Van Gogh} : \SI{1\,100\,000}{\euro} pour \(\,580\) élèves.\\[2pt]
Ces subventions sont-elles proportionnelles au nombre d’élèves ?

\bigskip
\section*{Exercice 26}
Dans la ville de Québec, une partie d’un édifice commercial est bâtie selon un modèle de pyramide à base carrée.\\[2pt]
Afin de respecter les différentes normes, la section pyramidale de cette bâtisse possède une base d’un périmètre de \SI{160}{\meter} et une hauteur de \SI{15}{\meter}.\\
Si \(70\%\) de cet espace est réservé à des bureaux administratifs, quel espace leur est alors consacré ? (préciser l’unité)

\bigskip
\section*{Exercice 27}
Dans un restaurant, on sert tous les breuvages dans des verres de même dimension.\\
Plus précisément, ces verres ont un rayon de \SI{7}{\centi\meter} et la partie qui peut contenir le liquide a une profondeur de \SI{8,5}{\centi\meter}.\\
Afin de bien fixer le prix des différents breuvages, déterminer, en \(\si{cm^3}\), le volume maximum de liquide que peut contenir un verre.

% ===========================
% PROPORTIONNALITE & VITESSES
% ===========================
\bigskip
\section*{Exercice 28}
Pour chaque tableau, calculer la \textbf{quatrième proportionnelle}.
\medskip

\renewcommand{\arraystretch}{1.35}
\begin{tabular}{@{}p{1cm}l@{}}
\textbf{a.} &
\(\displaystyle
\begin{array}{|c|c|}
\hline
152 & 1596\\
\hline
97 & x\\
\hline
\end{array}\) \\
[8pt]
\textbf{b.} &
\(\displaystyle
\begin{array}{|c|c|}
\hline
150 & 187{,}5\\
\hline
z & 28\\
\hline
\end{array}\) \\
[8pt]
\textbf{c.} &
\(\displaystyle
\begin{array}{|c|c|}
\hline
7 & 22\\
\hline
32{,}55 & y\\
\hline
\end{array}\) \\
[8pt]
\textbf{d.} &
\(\displaystyle
\begin{array}{|c|c|}
\hline
t & 147\\
\hline
29{,}8 & 365{,}05\\
\hline
\end{array}\) \\
\end{tabular}

\bigskip
\section*{Exercice 29}
\begin{enumerate}
  \item Monsieur Nomade roule à \SI{90}{\kilo\meter\per\hour}. Calculer, en minutes, le temps nécessaire pour parcourir \SI{36}{\kilo\meter}.
  \item Monsieur Nomade est parti à \SI{8}{h}. Il arrive à son entreprise à \SI{9}{h}\,20 en roulant à une vitesse moyenne de \SI{60}{\kilo\meter\per\hour}. Calculer, en kilomètres, la distance parcourue.
\end{enumerate}

\bigskip
\section*{Exercice 30}
Le pont d’Oléron est équipé d’un radar tronçon sur une distance de \SI{3,2}{\kilo\meter} et, sur le pont, la vitesse est limitée à \SI{90}{\kilo\meter\per\hour}.
\begin{enumerate}
  \item Monsieur Lagarde a mis \SI{2}{\minute} pour parcourir la distance entre les deux points d’enregistrement. Quelle est sa vitesse moyenne entre ces deux points ?
  \item La plaque d’immatriculation de Monsieur Durand a été enregistrée à \SI{13}{h}\,46\,\SI{54}{s} puis à \SI{13}{h}\,48\,\SI{41}{s}.\\
  Calculer sa vitesse moyenne lors de la traversée du pont.
\end{enumerate}

% ===========================
% STATISTIQUES
% ===========================

{\Large \textbf{Exercices — Statistiques et Probabilités}}\par\medskip

\bigskip
\section*{Exercice 31}
Suivant ses résultats scolaires, des parents donnent à leur enfant une somme différente d’argent de poche.
\medskip

\begin{tabular}{|c|cccccc|}
\hline
 & Janvier & Février & Mars & Avril & Mai & Juin\\
\hline
Somme (en \euro) & 30 & 28 & 25 & 45 & 15 & 22\\
\hline
\end{tabular}

\medskip
Quelle est la \textbf{somme moyenne mensuelle} qu’accordent les parents à leur enfant ?

\section*{Exercice 32}
Le loto\textendash Bingo est un jeu de tirage. La veille de Noël, un super tirage est organisé au cours duquel 350 joueurs participent. Le tableau ci-dessous indique la répartition des gains des gagnants ainsi que leur effectif.

\medskip
\renewcommand{\arraystretch}{1.25}
\begin{tabular}{|c|c|c|c|c|c|}
\hline
\textbf{Gain (en \euro)} & 2 & 10 & 50 & 100 & 1\,000\\
\hline
\textbf{Effectif} & 45 & 22 & 10 & 3 & 1\\
\hline
\end{tabular}

\medskip
\begin{enumerate}
  \item Calculer le \textbf{gain moyen des joueurs}. (arrondi au dixième)
  \item Calculer le \textbf{gain moyen des gagnants}. (arrondi au dixième)
\end{enumerate}

% ===========================
\section*{Exercice 33}
Un professeur de mathématiques calcule la moyenne de ses élèves de la manière suivante : \(70\%\) de la note correspond aux \textit{Devoirs Bilan}, \(20\%\) aux \textit{tests} et \(10\%\) à la \textit{participation}.

Voici les notes d’un de ses élèves :\\
\quad DB : \(12;\ 17;\ 9;\ 13\) \qquad
Test : \(16;\ 14\) \qquad
Participation : \(13\)

\medskip
Quelle est la \textbf{moyenne} de cet élève ? (arrondi au dixième)

\bigskip
{\Large \textbf{Exercices — Probabilités}}\par\medskip

% ===========================
% URNE
% ===========================
\section*{Exercice 34}
Une urne contient 7 boules rouges, 8 boules bleues et 5 boules vertes. On tire une boule au hasard.
\begin{enumerate}
  \item Quelle est la probabilité de tirer une boule verte ?
  \item Quelle est la probabilité de ne pas tirer une boule verte ?
\end{enumerate}
On tire à présent deux boules successivement et sans remise. La première boule tirée est rouge.
\begin{enumerate}\setcounter{enumi}{2}
  \item Quelle est la probabilité que la seconde boule tirée soit bleue ?
\end{enumerate}

% ===========================
% LOUVRE
% ===========================
\section*{Exercice 35}
Le musée du Louvre à Paris reçoit une matinée \(1\,250\) visiteurs :
\begin{itemize}
  \item \(550\) parisiens dont \(120\) parlent anglais ;
  \item \(450\) étrangers qui ne parlent qu’anglais ;
  \item les autres visiteurs viennent du reste de la France et \(80\) parlent anglais.
\end{itemize}
\begin{enumerate}
  \item Si je choisis un touriste pris au hasard dans le musée, quelle est la probabilité des événements suivants :
  \begin{enumerate}
    \item[A)] \(A\) : « le touriste est étranger » ;
    \item[B)] \(B\) : « le touriste vient du reste de la France et ne parle pas anglais » ;
    \item[C)] \(C\) : « le touriste parle anglais ».
  \end{enumerate}
  \item Si j’aborde un touriste dans ce musée, ai-je plus de chance de me faire comprendre en parlant \textbf{en anglais} ou \textbf{en français} ?
\end{enumerate}

% ===========================
% DÉ LETTRES
% ===========================
\section*{Exercice 36}
On écrit sur les faces d’un dé équilibré à six faces, chacune des lettres du mot : \texttt{NOTOUS}. On lance le dé et on regarde la lettre inscrite sur la face supérieure.
\begin{enumerate}
  \item Quelles sont les \textbf{issues} de cette expérience ?
  \item Déterminer la \textbf{probabilité} de chacun des événements suivants :
  \begin{enumerate}
    \item[E1)] « On obtient la lettre \(O\) » ;
    \item[E2)] \(E2\) est l’événement \textit{contraire} de \(E1\). Le décrire puis calculer sa probabilité ;
    \item[E3)] « On obtient une \textit{consonne} » ;
    \item[E4)] « On obtient une lettre du mot \texttt{KIWI} » ;
    \item[E5)] « On obtient une lettre du mot \texttt{CAGOUS} ».
  \end{enumerate}
\end{enumerate}

\end{document}
