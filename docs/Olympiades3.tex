\documentclass[11pt,a4paper]{article}

\usepackage[T1]{fontenc}
\usepackage[utf8]{inputenc}
\usepackage[french]{babel}
\usepackage{amsmath,amssymb}
\usepackage{enumitem}
\usepackage[margin=2.5cm]{geometry}

\usepackage{tikz}
\usetikzlibrary{calc}

\setlength{\parindent}{0pt}
\setlength{\parskip}{6pt}
\setlist[enumerate]{itemsep=4pt, topsep=4pt}

\begin{document}

\section*{Énoncé}

\subsection*{Partie A : Questions préliminaires}
On considère trois entiers deux à deux distincts et compris entre $1$ et $9$.
\begin{enumerate}
  \item Quelle est la plus petite valeur possible pour leur somme ?
  \item Quelle est la plus grande valeur possible pour leur somme ?
\end{enumerate}

\subsection*{Partie B : Les triangles magiques}
On place tous les nombres entiers de $1$ à $9$ dans les neuf cases situées sur le pourtour d'un triangle, comme indiqué sur la figure ci-dessous.

\begin{center}
\begin{tikzpicture}[scale=1, every node/.style={draw, minimum size=7mm, inner sep=1pt}]

  \coordinate (A) at (0,3.6);
  \coordinate (B) at (-3.6,0);
  \coordinate (C) at (3.6,0);

  \coordinate (L1) at ($(A)!1/3!(B)$);
  \coordinate (L2) at ($(A)!2/3!(B)$);

  \coordinate (R1) at ($(A)!1/3!(C)$);
  \coordinate (R2) at ($(A)!2/3!(C)$);

  \coordinate (D1) at ($(B)!1/3!(C)$);
  \coordinate (D2) at ($(B)!2/3!(C)$);


  \node at (A)  {$n_1$};
  \node at (L1) {$n_2$};
  \node at (L2) {$n_3$};
  \node at (B)  {$n_4$};
  \node at (D1) {$n_5$};
  \node at (D2) {$n_6$};
  \node at (C)  {$n_7$};
  \node at (R2) {$n_8$};
  \node at (R1) {$n_9$};


  \draw (A)--(B)--(C)--cycle;
\end{tikzpicture}
\end{center}

Si les sommes des quatre nombres situés sur chacun des trois côtés du triangle ont la même valeur $S$, on dit que le triangle est $S$-magique.
(C'est-à-dire si :
\[
n_1+n_2+n_3+n_4 \;=\; n_4+n_5+n_6+n_7 \;=\; n_7+n_8+n_9+n_1 \;=\; S.)
\]
On se propose de déterminer toutes les valeurs possibles de $S$.

\begin{enumerate}
  \item Compléter le triangle suivant de sorte qu'il soit $20$-magique, c'est-à-dire $S$-magique de somme $S=20$.
\end{enumerate}

\begin{center}
\begin{tikzpicture}[scale=1, every node/.style={draw, minimum size=7mm, inner sep=1pt}]
  \coordinate (A) at (0,3.6);
  \coordinate (B) at (-3.6,0);
  \coordinate (C) at (3.6,0);

  \coordinate (L1) at ($(A)!1/3!(B)$);
  \coordinate (L2) at ($(A)!2/3!(B)$);

  \coordinate (R1) at ($(A)!1/3!(C)$);
  \coordinate (R2) at ($(A)!2/3!(C)$);

  \coordinate (D1) at ($(B)!1/3!(C)$);
  \coordinate (D2) at ($(B)!2/3!(C)$);

  \node at (A)  {$2$};
  \node at (L1) {$n_2$};
  \node at (L2) {$n_3$};
  \node at (B)  {$5$};
  \node at (D1) {$n_5$};
  \node at (D2) {$n_6$};
  \node at (C)  {$8$};
  \node at (R2) {$n_8$};
  \node at (R1) {$n_9$};

  \draw (A)--(B)--(C)--cycle;
\end{tikzpicture}
\end{center}

\begin{enumerate}\setcounter{enumi}{1}
  \item On considère un triangle $S$-magique et on appelle $T$ la somme des nombres placés sur les trois sommets.
  \begin{enumerate}
    \item Prouver qu'on a $45+T=3S$.
    \item En déduire qu'on a $17 \le S \le 23$.
    \item Donner la liste des couples $(S,T)$ ainsi envisageables.
  \end{enumerate}
  \item Proposer un triangle $17$-magique.
  \item Prouver qu'il n'existe pas de triangle $18$-magique.
  \item
  \begin{enumerate}
    \item Montrer que dans un triangle $19$-magique, $7$ est nécessairement situé sur un sommet du triangle.
    \item Proposer un triangle $19$-magique.
  \end{enumerate}
  \item Prouver que, s'il existe un triangle $S$-magique, alors il existe aussi un triangle $(40-S)$-magique.
  \item Pour quelles valeurs de $S$ existe-t-il au moins un triangle $S$-magique ?
\end{enumerate}

\end{document}
