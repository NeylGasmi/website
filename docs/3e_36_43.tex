\documentclass[12pt,a4paper]{article}
\usepackage[utf8]{inputenc}
\usepackage[T1]{fontenc}
\usepackage[french]{babel}
\usepackage{amsmath,amssymb}
\usepackage{geometry}
\geometry{margin=2cm}

\begin{document}
Neyl Gasmi

\section*{Exercices 36 à 43}

\subsection*{Ex 36 \; Bambou brisé}
Lorsqu’il est brisé, un bambou de 1 mètre de hauteur a son extrémité qui touche le sol à une distance de 30 cm de sa base.

\medskip
À quelle hauteur a-t-il été brisé ?

\bigskip

\subsection*{Ex 37 \; Le lièvre et la tortue}
Un lièvre et une tortue font la course : ils s’élancent pour 5 km en ligne droite.  
Le lièvre court 5 fois plus vite que la tortue.

Au départ, le lièvre est parti par erreur perpendiculairement à la bonne route.  
Quand il s’en est aperçu, il a instantanément changé de direction pour aller tout droit vers l’arrivée.

Le lièvre et la tortue ont franchi l’arrivée exactement en même temps.  

\medskip
À quelle distance de l’arrivée se trouve le point où le lièvre a changé de direction ?

\bigskip

\subsection*{Ex 38 \; Les chariots}
Sur un parking de supermarché, se trouvent deux lignes de chariots bien rangés.

La première ligne, de 10 chariots, mesure 2,9 mètres de long.  
La seconde, de 20 chariots, mesure 4,9 mètres de long.  

\medskip
Quelle est la longueur d’un chariot ?

\bigskip

\subsection*{Ex 39 \; Équation \(x^2=a\)}
Résoudre les équations.

\textbf{Série 1}
\begin{enumerate}
  \item \(x^2 + 5 = 0\)
  \item \(2x^2 - 18 = 0\)
  \item \(x^2 - 1 = 2x^2 - 3\)
  \item \(3x^2 - 7 = x^2 - 7\)
\end{enumerate}

\textbf{Série 2}
\begin{enumerate}
  \item \((x-3)^2 = 49\)
  \item \((2x+7)^2 = 25\)
  \item \((x-4)(x+4) = 9\)
\end{enumerate}

\bigskip

\subsection*{Ex 40 \; Brevet 1996}
\begin{enumerate}
  \item Résoudre l’inéquation \(7x > 8x - 3\). Représenter les solutions sur une droite graduée.
  \item Résoudre l’inéquation \(-3x+1 > -5x-2\).
  \item Représenter sur une droite graduée les solutions du système :
  \[
  \begin{cases}
  7x > 8x-3 \\
  -3x+1 > -5x-2
  \end{cases}
  \]
\end{enumerate}

\bigskip

\subsection*{Ex 41 \; Fabrication sous contrainte}
Une entreprise de menuiserie fabrique 150 chaises par jour.  
Elle produit deux types de chaises, les unes vendues à 35 € pièce, les autres 60 € pièce.

L’entreprise souhaite que le montant des ventes soit strictement supérieur à 7 000 € par jour et elle veut fabriquer plus de chaises à 35 € que de chaises à 60 €.

\medskip
Combien doit-elle fabriquer de chaises à 35 € par jour ?

\bigskip

\subsection*{Ex 42}
ABC est un triangle isocèle de sommet principal A tel que \(AB=8\) cm et \(BC=9,6\) cm.  
On appelle respectivement H et K les pieds des hauteurs issues de A et C.

\begin{enumerate}
  \item Calculer AH puis l’aire du triangle ABC.
  \item Calculer CK puis BK.
\end{enumerate}

\bigskip

\subsection*{Ex 43 \; Brevet 1998}
ABC est un triangle tel que \(AB=4,2\) cm ; \(AC=5,6\) cm et \(BC=7\) cm.

\begin{enumerate}
  \item Démontrer que ABC est un triangle rectangle.
  \item Calculer son aire.
  \item On sait que si \(R\) est le rayon du cercle circonscrit à un triangle dont les côtés ont pour longueurs \(a, b, c\) données en cm, l’aire de ce triangle est égale à \(\dfrac{abc}{4R}\).
  \begin{enumerate}
    \item En utilisant cette formule, calculer le rayon du cercle circonscrit à ABC.
    \item Pouvait-on prévoir ce résultat ? Justifier la réponse.
  \end{enumerate}
\end{enumerate}

\end{document}
