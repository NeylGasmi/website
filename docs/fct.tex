\documentclass[11pt,a4paper]{article}

\usepackage[T1]{fontenc}
\usepackage[utf8]{inputenc}
\usepackage[french]{babel}
\usepackage{amsmath,amssymb}
\usepackage{enumitem}
\usepackage[margin=2.5cm]{geometry}
\usepackage{titlesec}
\newcounter{exercice}

\newcommand{\exo}[1]{%
  \refstepcounter{exercice}%
  \subsection*{Exercice \theexercice\ — #1}%
}


\titleformat{\subsection}
  {\normalfont\bfseries\small}
  {}
  {0pt}
  {}

\setlength{\parindent}{0pt}
\setlength{\parskip}{6pt}
\setlist[enumerate]{itemsep=4pt}

\begin{document}

\section*{Exercices}

\subsection*{Exercice 1 — Expression en fonction de \dots}

On considère $ABCD$ un rectangle, avec $AB=5$ et $BC=3$.  
Le point $M$ appartient au segment $[AB]$ avec $AM=x$.

\begin{enumerate}
  \item Quelles sont les valeurs que peut prendre la variable $x$ ?
  \item Donner l’expression de l’aire de $MBCD$ en fonction de $x$.
  \item Calculer l’aire de $MBCD$ lorsque $x=4$.
  \item Pour quelle valeur de $x$ l’aire de $MBCD$ est-elle égale au quart de l’aire de $ABCD$ ?
\end{enumerate}


\subsection*{Exercice 2 — Les trois traductions de $y=f(x)$}

Soit $f$ une fonction telle que $-1$ appartient à l’ensemble de définition $D_f$ de $f$.  
Donner trois traductions de la proposition :
\[
3=f(-1).
\]

\subsection*{Exercice 3 — Forme adéquate pour résoudre une équation}

On considère la fonction définie pour tout réel $x$ par
\[
f(x)=3x^2-12x.
\]

\begin{enumerate}
  \item Vérifier que $f(x)=3(x-2)^2-12$ (forme canonique de $f$).
  \item Factoriser $f(x)$.
  \item Résoudre l’équation $(E)$ : $f(x)=0$.
  \item Résoudre l’équation $(F)$ : $f(x)=-12$.
  \item Résoudre l’équation $(G)$ :
  \[
  f(x)=(x-4)(2x+1).
  \]
\end{enumerate}

\subsection*{Exercice 4 — Forme adéquate pour résoudre une équation}

On considère la fonction définie pour tout réel $x$ par
\[
f(x)=\frac{4x^2+3x-27}{x^2+9}.
\]

\begin{enumerate}
  \item Vérifier que
  \[
  f(x)=4+\frac{3x-63}{x^2+9}.
  \]
  \item Vérifier que
  \[
  f(x)=\frac{(4x-9)(x+3)}{x^2+9}.
  \]
  \item Calculer $f(0)$.
  \item Résoudre l’équation $f(x)=0$.
  \item Résoudre l’équation $f(x)=4$.
\end{enumerate}

\subsection*{Exercice 5 — Images, antécédents}

On considère la fonction $f$ définie pour $x\neq -2$ par
\[
f(x)=\frac{2x-5}{3x+6}.
\]

\begin{enumerate}
  \item Déterminer l’image de $2$ par $f$.
  \item Déterminer $f(-11)$.
  \item Déterminer le ou les antécédents par $f$ de $-\dfrac{1}{2}$.
  \item Résoudre l’équation $f(x)=\dfrac{2}{3}$.
\end{enumerate}

\subsection*{Exercice 6 — Tableau de valeurs, images, antécédents et point fixe}

On considère une fonction numérique $f$, dont les valeurs sont données par le tableau suivant :

\[
\begin{array}{c|ccccc}
x & -2 & 0 & 1 & 3 & 5\\ \hline
f(x) & 4 & 1 & 1 & -1 & 0
\end{array}
\]

\begin{enumerate}
  \item Quelle est l’image de $0$ par $f$ ?
  \item Le réel $0$ admet-il un antécédent par $f$ ?
  \item Existe-t-il un réel admettant plusieurs antécédents par $f$ ?
  \item On appelle \emph{point fixe} d’une fonction un réel $x$ tel que $f(x)=x$.  
  La fonction $f$ admet-elle des points fixes ?
\end{enumerate}

\subsection*{Problème — Le théorème du pâtissier}

On fabrique une boîte à partir d’une feuille de carton carrée de $12$ cm de côté.  
On découpe quatre carrés de $x$ cm de côté, un à chaque sommet.  
On note $V(x)$ le volume de la boîte exprimé en cm$^3$.  
On cherche le plus grand volume possible.

\begin{enumerate}
  \item Quel est l’ensemble de définition $D$ de $V$ ?
  \item Donner l’expression de $V$ sur $D$.
  \item Établir un tableau de valeurs pour $V$. Quelle conjecture peut-on émettre ?
  \item Démontrer que
  \[
  V(x)-V(2)=4(x-2)^2(x-8).
  \]
  Que peut-on en conclure ?
\end{enumerate}

\end{document}
