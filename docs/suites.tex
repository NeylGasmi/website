\documentclass[11pt,a4paper]{article}

\usepackage[T1]{fontenc}
\usepackage[utf8]{inputenc}
\usepackage[french]{babel}
\usepackage{amsmath,amssymb}
\usepackage{enumitem}
\usepackage[margin=2.5cm]{geometry}
\usepackage{titlesec}

% Style des titres d'exercices
\titleformat{\subsection}
  {\normalfont\bfseries\small}
  {}
  {0pt}
  {}

\setlength{\parindent}{0pt}
\setlength{\parskip}{6pt}
\setlist[enumerate]{itemsep=4pt}

\begin{document}

\section*{Exercices}

\subsection*{Exercice 1}

On définit trois suites $(c_n)_{n\in\mathbb{N}}$, $(s_n)_{n\in\mathbb{N}}$ et $(t_n)_{n\in\mathbb{N}}$ par
\[
\forall n\in\mathbb{N},\quad
c_n=\frac{2^n+2^{-n}}{2},\qquad
s_n=\frac{2^n-2^{-n}}{2},\qquad
t_n=\frac{s_n}{c_n}.
\]

\begin{enumerate}
  \item Démontrer que ces trois suites sont croissantes.
  \item Démontrer que, pour tout $(n,m)\in\mathbb{N}^2$,
  \[
  c_{n+m}=c_nc_m+s_ns_m,\quad
  s_{n+m}=s_nc_m+s_mc_n,\quad
  t_{n+m}=\frac{t_n+t_m}{1+t_nt_m}.
  \]
  \item Démontrer que, pour tout $n\in\mathbb{N}$,
  \[
  c_n^2-s_n^2=1.
  \]
\end{enumerate}

\subsection*{Exercice 2}

On considère la suite $(u_n)$ définie par
\[
\forall n\in\mathbb{N},\quad
u_n=\frac{2n^2+1}{n^2+5}.
\]

Montrer que $(u_n)$ est une suite strictement croissante sur $\mathbb{N}$.

\subsection*{Exercice 3}

On considère la suite $(u_n)$ définie par
\[
u_n=1+\frac{1}{2^2}+\frac{1}{3^2}+\cdots+\frac{1}{n^2}.
\]

Montrer que $(u_n)$ est strictement croissante.

\subsection*{Exercice 4}

Étudier le sens de variation de la suite $(u_n)$ définie pour tout entier naturel $n$ par :
\begin{enumerate}
  \item $u_n=n^2+4n+3$,
  \item $u_n=\dfrac{2^n}{n+1}$,
  \item $u_n=\dfrac{1-n^2}{n+2}$,
  \item $u_0=1$ et $u_{n+1}=u_n-\dfrac{1}{n+1}$.
\end{enumerate}

\subsection*{Exercice 5 — Suite arithmético-géométrique}

Soit $a\neq 1$ et $b$ deux réels.  
On définit la suite arithmético-géométrique $(u_n)$ par
\[
u_0\in\mathbb{R},\qquad \forall n\in\mathbb{N},\quad u_{n+1}=au_n+b.
\]

\begin{enumerate}
  \item Que se passe-t-il si $b=0$ ?
  \item Soit $x_0$ l’unique solution de l’équation $ax+b=x$.  
  Que se passe-t-il si $u_0=x_0$ ?
  \item Soit $(v_n)$ la suite définie par
  \[
  \forall n\in\mathbb{N},\quad v_n=u_n-x_0.
  \]
  Démontrer que $(v_n)$ est géométrique de raison $a$ et préciser $v_0$.
  \item À partir de la forme explicite de $(v_n)$, donner celle de $(u_n)$.
  \item Application numérique : $a=-2$, $b=3$ et $u_0=-1$.
\end{enumerate}

\subsection*{Exercice 6}

Soit $(u_n)$ une suite vérifiant
\[
\forall n\ge 0,\quad u_{n+1}=2u_n+n \quad \text{et} \quad u_0=1.
\]

\begin{enumerate}
  \item Montrer qu’il existe $(a,b)\in\mathbb{R}^2$ tel que la suite $(w_n)$ définie par
  \[
  \forall n\in\mathbb{N},\quad w_n=an+b
  \]
  vérifie $w_{n+1}=2w_n+n$.
  \item Montrer que la suite $z_n=u_n+n+1$ vérifie $z_{n+1}=2z_n$.
  \item En déduire l’expression de $z_n$ puis celle de $u_n$.
\end{enumerate}

\subsection*{Exercice 7}

On considère l’ensemble $(E)$ des suites $(x_n)$ définies sur $\mathbb{N}$ et vérifiant
\[
\forall n\in\mathbb{N}^*,\quad x_{n+1}-x_n=0{,}24\,x_{n-1}.
\]

\begin{enumerate}
  \item Soit $\lambda\neq 0$ et $t_n=\lambda^n$.  
  Montrer que $(t_n)\in(E)$ si et seulement si $\lambda$ est solution de
  \[
  \lambda^2-\lambda-0{,}24=0.
  \]
  En déduire les suites $(t_n)$ appartenant à $(E)$.
  \item On admet que $(E)$ est l’ensemble des suites $(u_n)$ définies par
  \[
  \exists (\alpha,\beta)\in\mathbb{R}^2,\quad
  \forall n\in\mathbb{N},\quad
  u_n=\alpha(1{,}2)^n+\beta(-0{,}2)^n.
  \]
  Déterminer $\alpha$ et $\beta$ sachant que $u_0=6$ et $u_1=6{,}6$.
\end{enumerate}

\subsection*{Exercice 8}

Soit $(u_n)$ la suite définie pour $n\ge 1$ par
\[
\forall n\in\mathbb{N},\quad
u_n=\sum_{k=1}^n \frac{1}{\sqrt{k}}
=1+\frac{1}{\sqrt{2}}+\frac{1}{\sqrt{3}}+\cdots+\frac{1}{\sqrt{n}}.
\]

\begin{enumerate}
  \item Justifier que, pour tout $k$ tel que $1\le k\le n$, on a
  \[
  \frac{1}{\sqrt{k}}\ge\frac{1}{\sqrt{n}}.
  \]
  \item En déduire que, pour tout $n\ge 1$, $u_n\ge\sqrt{n}$.
\end{enumerate}

\subsection*{Problème — Suite homographique}

On considère la suite $(u_n)$ définie par
\[
u_0=1,\qquad \forall n\in\mathbb{N},\quad
u_{n+1}=\frac{3+2u_n}{2+u_n}.
\]

\begin{enumerate}
  \item Calculer les quatre premiers termes de la suite.
  \item Expliquer pourquoi tous les termes de $(u_n)$ sont positifs.
  \item Démontrer que, pour tout $n\in\mathbb{N}^*$,
  \[
  u_n\ge \frac{3}{2}.
  \]
  \item On définit $(v_n)$ par
  \[
  \forall n\in\mathbb{N},\quad
  v_n=\frac{u_n-\sqrt{3}}{u_n+\sqrt{3}}.
  \]
  \begin{enumerate}
    \item Montrer que $(v_n)$ est géométrique (préciser $v_0$ et la raison).
    \item Exprimer $v_n$ puis $u_n$ en fonction de $n$.
    \item Étudier les variations de $(u_n)$.
  \end{enumerate}
  \item Démontrer que, pour tout $n\in\mathbb{N}$,
  \[
  u_n<\sqrt{3}.
  \]
\end{enumerate}

\end{document}
