\documentclass[11pt,a4paper]{article}

\usepackage[T1]{fontenc}
\usepackage{titlesec}
\usepackage[utf8]{inputenc}
\usepackage[french]{babel}
\usepackage{amsmath,amssymb}
\usepackage{enumitem}
\usepackage[margin=2.5cm]{geometry}

\setlength{\parindent}{0pt}
\setlength{\parskip}{6pt}
\setlist[enumerate]{itemsep=4pt}
\titleformat{\subsection}
  {\normalfont\bfseries}
  {}
  {0pt}
  {}

\begin{document}

\section*{Quiz}

\begin{enumerate}
  \item La droite $D$ a pour équation $x=3$. Est-ce une droite horizontale ? Verticale ?
  \item Quelle est l'équation réduite d'une droite horizontale ?
  \item Si $D : y=ax+b$, comment sont appelés $a$ et $b$ ?
  \item On sait que la droite $D$ passe par $A(x_A,y_A)$ et $B(x_B,y_B)$. Combien vaut son coefficient directeur ?
  \item Comment déterminer l'équation de la droite passant par $A(x_A,y_A)$ et $B(x_B,y_B)$ ?
  \item À quelle condition deux droites sont-elles parallèles ?
  \item La droite $D$ a pour équation $y=4x-5$. Le point $A(1,0)$ appartient-il à $D$ ?
  \item Les droites $D : y=\frac13x-1$ et $\Delta : y=2+\frac13x$ sont-elles parallèles ?
  \item $D : y=ax+b$ et $\Delta : y=mx+p$. Si $D$ et $\Delta$ sont sécantes, quel système leur point d'intersection vérifie-t-il ?
  \item Quelle peut être l'intersection de deux droites ?
\end{enumerate}

\section*{Exercice 1 — Donner l’équation d’une droite avec deux points}

Dans les cas suivants, donner l’équation réduite de la droite $(AB)$.
\begin{enumerate}
  \item $A(2,1)$ et $B(-1,4)$.
  \item $A(-3,2)$ et $B(5,-1)$.
  \item $A(1,0)$ et $C(5,4)$ et $B$ est le milieu de $[AC]$.
\end{enumerate}

\section*{Exercice 2 — Deux problèmes voisins}

\begin{enumerate}
  \item Soit $A(-3,4)$ et $B(2,1)$. Déterminer une équation de la droite $(AB)$.
  \item Soit $A(2,3)$ et $B(-1,5)$. Démontrer que $2x+3y=13$ est une équation de la droite $(AB)$.
\end{enumerate}

\section*{Exercice 3 — Droite parallèle}

Dans les cas suivants, donner une équation de la droite demandée.
\begin{enumerate}
  \item La droite $D$ passant par $A(2,1)$ et parallèle à $\Delta : y=-3x+1$.
  \item La droite $D$ passant par $A(5,0)$ et parallèle à $\Delta : y=12x+2$.
\end{enumerate}

\section*{Exercice 4 — Droites, équations et coefficients directeurs}

Pour chacune des équations de droites suivantes, exprimées dans un repère orthonormal, déterminer :
\begin{itemize}
  \item la nature de la droite (oblique, horizontale ou verticale),
  \item le coefficient directeur,
  \item un point appartenant à la droite.
\end{itemize}

\begin{enumerate}
  \item $y=\dfrac{7x-2}{3}$
  \item $3x-4y=1$
  \item $\dfrac{x}{3}-\dfrac{y}{5}=1$
  \item $5x=2$
  \item $x-y=x-1$
\end{enumerate}

\section*{Exercice 6 — Déjà vu}

Dans un repère, les points $A(0,-7)$, $B(-4,0)$ et $C(77,-146)$ sont-ils alignés ?

\section*{Exercice 7 — Points alignés et équations de droites}

Soit le repère orthonormal $(O;I;J)$.  
On considère les points $A(3,0)$, $B(0,3)$, $C(7,0)$, $D(0,7)$.  
On appelle $E$ le milieu du segment $[AB]$, $F$ le milieu du segment $[DC]$, et $G$ le point d’intersection des droites $(AD)$ et $(BC)$.

\begin{enumerate}
  \item Déterminer les équations réduites des droites $(AD)$ et $(BC)$.
  \item En déduire les coordonnées du point $G$.
  \item Calculer les coordonnées des points $E$ et $F$.
  \item Conclure que les points $O$, $E$, $G$, $F$ sont alignés.
\end{enumerate}

\section*{Exercice 9 — Avec un paramètre}

Résoudre, suivant les valeurs de $t\in\mathbb{R}$, le système d’inconnue $(x,y)\in\mathbb{R}^2$ :
\[
\begin{cases}
2tx+(t+1)y=2\\
(t+2)x+(2t+1)y=t+2
\end{cases}
\]

\section*{Exercice 10 — Coefficient directeur}

Soient $A(x_A,y_A)$ et $B(x_B,y_B)$ avec $x_A\neq x_B$.  
Démontrer que le coefficient directeur de la droite $(AB)$ est
\[
m=\dfrac{y_B-y_A}{x_B-x_A}.
\]

\section*{Exercice 11 — Deux représentations paramétriques}

Soient
\[
(D)\ :
\begin{cases}
x=-1+4t\\
y=2-3t
\end{cases}
\qquad
(D')\ :
\begin{cases}
x=1-t\\
y=3+5t
\end{cases}
\quad (t\in\mathbb{R})
\]

\begin{enumerate}
  \item Les vecteurs $\vec u(4,-3)$ et $\vec v(-1,5)$ sont-ils colinéaires ?  
  Que peut-on en déduire pour l’intersection des droites $(D)$ et $(D')$ ?
  \item Démontrer que le système
  \[
  \begin{cases}
  -1+4t=1-t\\
  2-3t=3+5t
  \end{cases}
  \]
  n’admet pas de solution. Comment concilier ce résultat avec le précédent ?
  \item Déterminer l’intersection des droites $(D)$ et $(D')$.
\end{enumerate}

\section*{Exercice 12}

Soit $D$ une droite du plan qui ne passe pas par l’origine.  
On suppose que $D$ coupe l’axe des abscisses en $A(\alpha,0)$ et l’axe des ordonnées en $B(0,\beta)$.  
Donner une équation de la droite $D$.

\section*{Exercice 14 — Suite arithmético-géométrique}

Soit la suite $(u_n)$ définie par $u_0$ et :
\[
\forall n\in\mathbb{N},\quad u_{n+1}=f(u_n)\quad \text{où}\quad f(x)=ax+b.
\]

Pour chacune des valeurs de $a$ et $b$ données, répondre aux questions ci-dessous :
\[
\text{i) } a=\tfrac12,\ b=1
\qquad
\text{ii) } a=-\tfrac23,\ b=5
\qquad
\text{iii) } a=1{,}5,\ b=-1
\]

\begin{enumerate}
  \item Représenter la courbe de $f$ dans un repère orthonormal ainsi que la droite $D$ d’équation $y=x$.
  \item Représenter les premiers termes de la suite sur l’axe des abscisses.
  \item Déterminer $\alpha$, solution de $f(x)=x$.
  \item On pose $v_n=u_n-\alpha$. Démontrer que $(v_n)$ est une suite géométrique.
  \item En déduire l’expression de $u_n$ en fonction de $n$.
\end{enumerate}

\end{document}
