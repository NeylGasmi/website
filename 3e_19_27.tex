\documentclass[a4paper,11pt]{article}
\usepackage[utf8]{inputenc}
\usepackage[T1]{fontenc}
\usepackage[french]{babel}
\usepackage{amsmath,amssymb}
\usepackage{geometry}
\geometry{margin=2cm}
\setlength{\parindent}{0pt}

\begin{document}
Neyl Gasmi

\section*{Exercices 19 à 27}


\subsection*{Ex 19 \; Distributivité simple}
Développer et simplifier.

\textbf{Série 1}
\[
\begin{aligned}
\mathrm{A}&=a(a+3) &\quad \mathrm{F}&=b(3b^{2}+5b)\\
\mathrm{B}&=(x^{2}+4)x &\quad \mathrm{G}&=(7b^{2}-6b)b\\
\mathrm{C}&=a(a^{2}+a) &\quad \mathrm{H}&=x(5x^{2}+2x)\\
\mathrm{D}&=a(a^{2}+1) &\quad \mathrm{I}&=(a^{2}+2a)a\\
\mathrm{E}&=a(3a+6)
\end{aligned}
\]

\textbf{Série 2}
\[
\begin{aligned}
\mathrm{A}&=x^{2}(2x+x^{2}) &\quad \mathrm{D}&=(3a-9a^{2})a^{2}\\
\mathrm{B}&=(2x^{2}-5)\,3x^{2} &\quad \mathrm{E}&=5x^{2}(8x-9)\\
\mathrm{C}&=7a^{3}(3a^{3}+2a) &\quad \mathrm{F}&=3x(5x^{2}-3x)
\end{aligned}
\]

\textbf{Série 3}
\[
\begin{aligned}
\mathrm{A}&=2xy(x^{2}y+x) &\quad \mathrm{D}&=(3a^{3}-2a^{2}b-1)\,4ab\\
\mathrm{B}&=(2ab-4ab^{2})\,3a^{2}b &\quad \mathrm{E}&=2x^{3}(3xy+x)\\
\mathrm{C}&=5y^{2}(y^{3}-2x^{2}y+1) &\quad \mathrm{F}&=(2a^{2}b-3b)\,ab
\end{aligned}
\]


\subsection*{Ex 20 \; Double distributivité}
Développer et simplifier.

\textbf{Série 1}
\[
\begin{aligned}
\mathrm{A}&=(3t-2)(7t-4) &\quad \mathrm{D}&=(7y-3)(2y-1)\\
\mathrm{B}&=(4s-1)(2s+5) &\quad \mathrm{E}&=(x+3y)(2x-y)\\
\mathrm{C}&=(3x+5)(2x-1) &\quad \mathrm{F}&=(4x-5y)(x-y)
\end{aligned}
\]

\textbf{Série 2}
\[
\begin{aligned}
\mathrm{A}&=(3x^{2}-5)(2x^{2}+1) &\quad \mathrm{D}&=(3y^{2}-5x)(3x+5y^{2})\\
\mathrm{B}&=(a^{2}b+3a)(2a^{2}b-a) &\quad \mathrm{E}&=(2x^{2}-3x)(-4x+5x^{2})\\
\mathrm{C}&=(5ab-2b)(ab-4b) &\quad \mathrm{F}&=(-2x^{2}-5y)(-x-4y^{2})
\end{aligned}
\]

\textbf{Série 3}
\[
\begin{aligned}
\mathrm{A}&=(2a^{3}-7b)(-7a+3b^{2})\\
\mathrm{B}&=(5abc-2ab)(12ab-15abc)\\
\mathrm{C}&=(5ab^{2}+3a^{2}b)(-4a^{2}b+3ab^{2})\\
\mathrm{D}&=(2a^{3}b-7ab^{3})(-a^{3}b+2ab^{3})
\end{aligned}
\]


\subsection*{Ex 21 \; Identités remarquables}
Développer et simplifier.

\textbf{Série 1}
\[
\begin{aligned}
\mathrm{A}&=(7x-2y)^{2} &\quad \mathrm{D}&=(7x-12y)^{2}\\
\mathrm{B}&=(4a-2b)^{2} &\quad \mathrm{E}&=(2b-7c)^{2}\\
\mathrm{C}&=(3a+2b)^{2} &\quad \mathrm{F}&=(x-7y)(x+7y)
\end{aligned}
\]

\textbf{Série 2}
\[
\begin{aligned}
\mathrm{A}&=(3a-2b)^{2} &\quad \mathrm{C}&=(6a+b)^{2}\\
\mathrm{B}&=(2-2b)^{2} &\quad \mathrm{D}&=(3x-z)(3x+z)\\
& &\quad \mathrm{E}&=(4a-7)^{2}\\
& &\quad \mathrm{F}&=(10a-7b)^{2}
\end{aligned}
\]

\textbf{Série 3}
\[
\begin{aligned}
\mathrm{A}&=(2a-b^{2})^{2} &\quad \mathrm{D}&=(3a^{2}-2b^{3})^{2}\\
\mathrm{B}&=(2a^{2}+b)^{2} &\quad \mathrm{E}&=(x^{3}+y^{3})(x^{3}-y^{3})\\
\mathrm{C}&=(3x^{2}-y)(3x^{2}+y) &\quad \mathrm{F}&=\left(3x^{2}-\tfrac{1}{3}x\right)^{2}
\end{aligned}
\]


\subsection*{Ex 22 \; Identités enchaînées}
Développer astucieusement et simplifier.

\textbf{Série 1}
\[
\begin{aligned}
\mathrm{A}&=(x+a)(x-a)(x^{2}-a^{2}) &\quad \mathrm{D}&=(x+2)(x-2)(x^{4}+16)(x^{2}+4)\\
\mathrm{B}&=(2a-1)(2a+1)(4a^{2}+1) &\quad \mathrm{E}&=(x^{2}-1)(x^{2}+1)(x^{4}-8)\\
\mathrm{C}&=(x-1)(x^{2}+1)(x+1) &\quad \mathrm{F}&=(4a^{4}+3)(2a^{2}+1)(2a^{2}-1)
\end{aligned}
\]

\textbf{Série 2}
\[
\begin{aligned}
\mathrm{A}&=((x-1)+x^{2})((x-1)-x^{2})\\
\mathrm{B}&=(x+(2+x^{2}))(x-(2+x^{2}))\\
\mathrm{C}&=(x+y-1)(x-y+1)\\
\mathrm{D}&=(a^{2}-ab+b^{2})(a^{2}+ab+b^{2})
\end{aligned}
\]


\subsection*{Ex 23}
Soient \(x,y,z\) trois nombres réels non nuls. On pose :
\[
a=\frac{y}{z}+\frac{z}{y},\qquad
b=\frac{z}{x}+\frac{x}{z},\qquad
c=\frac{x}{y}+\frac{y}{x}.
\]
Calculer \(a^{2}+b^{2}+c^{2}-abc\).


\subsection*{Ex 24}
Sachant que \(X+Y=1\) et \(X^{2}+Y^{2}=2\),
\[
\text{1) que vaut }XY?\qquad
\text{2) que vaut } \frac{1}{X}+\frac{1}{Y} ?\qquad
\text{3) que vaut }X^{3}+Y^{3}?\qquad
\text{4) que vaut }X^{4}+Y^{4}?
\]


\subsection*{Ex 25 \; Facteurs communs}
Factoriser en utilisant un facteur commun.
\[
\begin{aligned}
\mathrm{A}&=3(x-2)+(x+3)(x-2)\\
\mathrm{B}&=5x(x-3)-x(2x+1)\\
\mathrm{C}&=(x+5)^{2}+(x-5)(x+5)-3(x+5)\\
\mathrm{D}&=5(2x-1)^{3}+(2x-1)^{2}(x+2)\\
\mathrm{E}&=x^{2}(x-2)+3x^{3}\\
\mathrm{F}&=(x-3)^{2}-2x(x-3)+(x-3)\\
\mathrm{G}&=(2x-3)^{2}+5x(3-2x)\\
\mathrm{H}&=(2x+5)+(x+3)(4x+10)\\
\mathrm{I}&=(x+9)(x-5)+2(6x-30)
\end{aligned}
\]


\subsection*{Ex 26 \; Identités remarquables}
Factoriser en utilisant une identité remarquable.

\textbf{Série 1}
\[
\begin{aligned}
\mathrm{A}&=x^{2}+10x+25 &\quad \mathrm{E}&=16x^{2}-8x+1\\
\mathrm{B}&=16-25x^{2} &\quad \mathrm{F}&=64-(2x+3)^{2}\\
\mathrm{C}&=1-12x+36x^{2} &\quad \mathrm{G}&=(3x-1)^{2}-9\\
\mathrm{D}&=(x+7)^{2}-1 &\quad \mathrm{H}&=4x^{2}-20x+25
\end{aligned}
\]

\textbf{Série 2}
\[
\begin{aligned}
\mathrm{A}&=4x^{2}-(x-5)^{2} &\quad \mathrm{D}&=9(x+1)^{2}-36\\
\mathrm{B}&=\tfrac14 x^{2}+x+1 &\quad \mathrm{E}&=(2x+3)^{2}-(x-1)^{2}\\
\mathrm{C}&=81+4x^{2}+36x &\quad \mathrm{F}&=\tfrac{4}{9}-\left(2x+\tfrac13\right)^{2}
\end{aligned}
\]

\textbf{Série 3}
\[
\begin{aligned}
\mathrm{A}&=x^{2}-9+(x-3)(2x+5)\\
\mathrm{B}&=5x(4x-1)+16x^{2}-1\\
\mathrm{C}&=x^{2}-25+x-5\\
\mathrm{D}&=4x^{2}+4x+1-(2x+1)(3-5x)
\end{aligned}
\]


\subsection*{Ex 27 \; Regroupements de termes}
Factoriser aussi complètement que possible.

\textbf{Série 1}
\[
\begin{aligned}
\mathrm{A}&=ax+ay+bx+by\\
\mathrm{B}&=ab+ac+bd+dc\\
\mathrm{C}&=ad+ac-bd-bc\\
\mathrm{D}&=21xy-3x-28y+4\\
\mathrm{E}&=ac+3ad-2bc-6bd\\
\mathrm{F}&=5ax-5ay-bx+by
\end{aligned}
\]

\textbf{Série 2}
\[
\begin{aligned}
\mathrm{A}&=x^{3}+4x^{2}+x+4\\
\mathrm{B}&=3x^{3}-x^{2}+6x-2\\
\mathrm{C}&=5x^{3}+x^{2}+5x+1\\
\mathrm{D}&=18x^{3}+9x^{2}+2x+1\\
\mathrm{E}&=x^{3}+x^{2}+x+1\\
\mathrm{F}&=x^{5}+x^{4}+x+1
\end{aligned}
\]

\end{document}
