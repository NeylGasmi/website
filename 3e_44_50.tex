\documentclass[a4paper,11pt]{article}
\usepackage[utf8]{inputenc}
\usepackage[T1]{fontenc}
\usepackage[french]{babel}
\usepackage{amsmath,amssymb}
\usepackage{geometry}
\geometry{margin=2cm}
\setlength{\parindent}{0pt}

\begin{document}
Neyl Gasmi
\begin{center}
{\Large \textbf{Exercices 44 à 50}}
\end{center}
\bigskip

\section*{Ex 44 \quad Brevet 2017}
Voici un programme de calcul :
\begin{itemize}
  \item choisir un nombre ;
  \item ajouter 1 à ce nombre ;
  \item calculer le carré du résultat ;
  \item soustraire le carré du nombre de départ au résultat précédent ;
  \item écrire le résultat.
\end{itemize}
\begin{enumerate}
  \item On choisit \(4\) comme nombre de départ. Quel est le résultat obtenu ?
  \item On note \(x\) le nombre choisi. Exprimer le résultat du programme en fonction de \(x\).
  \item Soit \(f\) la fonction définie par \(f:x\mapsto 2x+1\).
  \begin{enumerate}
    \item Calculer l’image de \(0\) par \(f\).
    \item Déterminer l’antécédent de \(5\) par \(f\).
  \end{enumerate}
\end{enumerate}

\section*{Ex 45}
\begin{enumerate}
  \item \(f\) est une fonction affine telle que \(f(2)=4\) et \(f(3)=9\). Combien vaut \(f(4)\) ?
  \item Soient \(f\) une fonction linéaire et \(g\) une fonction affine telles que
  \[
     f(2)=g(2)=4 \qquad \text{et} \qquad f(3)=-g(3).
  \]
  Combien vaut \(g(1)\) ?
\end{enumerate}

\section*{Ex 46}
Soit \(f:x\mapsto 2015\,x^{17}-2\).
On considère le nombre \(h\) (que l’on ne cherchera pas à calculer) tel que \(f(h)=-2015\).
Calculer \(f(-h)\).

\section*{Ex 47}
Soit \(f\) une fonction telle que \(f(1)=\dfrac12\) et
\[
  f(x+y)=f(x)\,f(y)\quad\text{pour tous entiers }x\text{ et }y.
\]
Combien vaut \(f(2)+f(0)+f(2)+f(1)\) ?

\section*{Ex 48 \quad Triangle}
Démontrer qu’un triangle (non aplati) dont les côtés sont des nombres entiers et dont le périmètre vaut \(8\), est isocèle.

\section*{Ex 49 \quad Le ballon de foot}
Un ballon de football est formé de \(12\) pentagones réguliers et de \(20\) hexagones réguliers assemblés entre eux par une couture.
Leurs côtés mesurent \(4{,}5\) cm.
\medskip

Trouver la longueur totale de la couture.

\section*{Ex 50 \quad Trains}
Deux trains, \(A\) et \(B\), suivent des parcours circulaires et se croisent à la gare \(G\).
Le train \(A\) fait un tour en \(5\) minutes.
Le train \(B\) fait un tour en \(7\) minutes.
\medskip

Dans combien de temps les deux trains se retrouveront-ils ensemble à la gare
\begin{enumerate}
  \item s’ils sont partis en même temps de la gare ?
  \item si le train \(A\) est parti de la gare depuis \(4\) minutes et le train \(B\) depuis \(2\) minutes ?
\end{enumerate}

\end{document}
