\documentclass[a4paper,11pt]{article}
\usepackage[utf8]{inputenc}
\usepackage[T1]{fontenc}
\usepackage[french]{babel}
\usepackage{amsmath,amssymb}
\usepackage{geometry}
\geometry{margin=2cm}
\setlength{\parindent}{0pt}

% TikZ pour la figure du dé (couleurs incluses via xcolor)
\usepackage{tikz}

\begin{document}
Neyl Gasmi
\begin{center}
{\Large \textbf{Exercices 51 à 53}}
\end{center}
\bigskip

% --- Ex 55 ---
\section*{Ex 51 \quad Probabilités}
Trois nombres différents sont choisis au hasard parmi \(2, 0, 2, 1\).  

Quelle est la probabilité d’obtenir \(0\) comme résultat de la multiplication des trois nombres ?


\section*{Ex 52 \quad Probabilités}
On lance deux dés (les dés sont des dés équilibrés standard à six faces marquées de \(1\) à \(6\)).  
On calcule la somme des deux nombres obtenus.  

Quelle est la probabilité que cette somme soit un nombre premier ?

% --- Ex 58 ---
\section*{Ex 53 \quad Mensonges}
Un garçon dit toujours la vérité le jeudi et le vendredi.  
Il ment toujours le mardi.  
Les autres jours, il ment ou dit la vérité au hasard.  

Sept jours de suite, on lui demande son prénom. Voici, dans l’ordre, ses réponses des six premiers jours :  
\emph{John, Bob, John, Bob, Pit, Bob.}  

Quelle est sa réponse le septième jour ?

\end{document}
