\documentclass[12pt,a4paper]{article}
\usepackage[utf8]{inputenc}
\usepackage[T1]{fontenc}
\usepackage[french]{babel}
\usepackage{amsmath,amssymb}
\usepackage{lmodern}
\usepackage{geometry}
\geometry{margin=2.5cm}

\title{Exercices de récurrence}
\author{}
\date{}

\begin{document}

\maketitle

\section*{Exercice 1}
On définit la suite $(u_n)_{n \in \mathbb{N}^*}$ par $u_1=1$ et la relation valable pour tout $n \in \mathbb{N}^*$ :
\[
u_{n+1} = u_n + 2n + 1.
\]
\begin{enumerate}
\item Calculer $u_2, u_3, u_4$.
\item Quelle conjecture peut-on faire sur l’expression de $u_n$ en fonction de $n$ ?
\item Prouver que si pour un certain $k \in \mathbb{N}^*$ on a $u_k = k^2$, alors $u_{k+1} = (k+1)^2$.
\item Conclure et calculer : $1+3+5+7+\cdots+2019$.
\end{enumerate}

\section*{Exercice 2}
Prouver par récurrence que pour tout $n \in \mathbb{N}$, l’entier $8^n - 1$ est divisible par 7.

\section*{Exercice 3}
Prouver par récurrence que pour tout $n \in \mathbb{N}^*$ :
\[
\sigma_n = 1+2+3+\cdots+n = \frac{n(n+1)}{2}.
\]

\section*{Exercice 4}
\begin{enumerate}
\item Justifier que pour tout $n \in \mathbb{N}^*$, on a $\sqrt{n^2+n} \geq n$.
\item En déduire par récurrence que
\[
\sum_{k=1}^n \frac{1}{\sqrt{k}} \geq \sqrt{n}.
\]
\end{enumerate}

\section*{Exercice 5}
D’après l’inégalité triangulaire :
\[
\forall x,y \in \mathbb{R}, \quad |x+y| \leq |x|+|y|.
\]
Démontrer que pour tout entier $n \geq 2$,
\[
\forall x_1, \dots, x_n \in \mathbb{R}, \quad 
\left| \sum_{k=1}^n x_k \right| \leq \sum_{k=1}^n |x_k|.
\]

\section*{Exercice 6}
Prouver que la suite de terme général
\[
u_n = \frac{2^n}{n!}
\]
est décroissante sur $\mathbb{N}^*$.


\section*{Exercice 7}
Écrire le produit des $n$ premiers entiers pairs non nuls avec la notation factorielle :
\[
2 \times 4 \times 6 \times \cdots \times 2n = ?
\]

\section*{Exercice 8}
Prouver par récurrence que pour $n$ entier assez grand,
\[
2^n \geq n^2.
\]

\section*{Exercice 9}
Soit $n$ un entier naturel et $\mathcal{P}(n)$ la proposition :
\[
\sum_{k=0}^n k = \tfrac{1}{2}\left(n+\tfrac{1}{2}\right)^2.
\]
Démontrer que $\mathcal{P}(n)$ est héréditaire, c’est-à-dire
\[
\forall n \in \mathbb{N}^*, \quad \mathcal{P}(n-1) \implies \mathcal{P}(n).
\]

\section*{Exercice 10}
Soit $n \geq 1$. On pose
\[
u_n = 1+2+3+\cdots+n = \frac{n(n+1)}{2}, \quad
v_n = 1^2+2^2+3^2+\cdots+n^2, \quad
q_n = \frac{v_n}{u_n}.
\]
\begin{enumerate}
\item Écrire un programme qui calcule pour tout $n \geq 1$, $v_n$ et $q_n$.  
Écrire un programme qui calcule pour tout $n \geq 1$, $q_{n+1}-q_n$.
\item Faire tourner ce programme pour diverses valeurs de $n$. Que constate-t-on ?
\item En supposant que la constatation soit générale, qu’en déduit-on pour la suite $(q_n)$ et pour la suite $(v_n)$ ?
\item Démontrer par récurrence la formule explicite trouvée pour $(v_n)$.
\end{enumerate}

\end{document}
