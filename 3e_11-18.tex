\documentclass[a4paper,11pt]{article}
\usepackage[utf8]{inputenc}
\usepackage[T1]{fontenc}
\usepackage[french]{babel}
\usepackage{amsmath,amssymb}
\usepackage{geometry}
\geometry{margin=2cm}
\setlength{\parindent}{0pt}

\begin{document}
Neyl Gasmi
\section*{Exercices 11 à 18}

\subsection*{Ex 11}
En ajoutant \(4^{15}\) et \(8^{10}\), on obtient une puissance de \(2\). Laquelle ?

\bigskip

\subsection*{Ex 12}
Dans chacun des cas, déterminer l’entier naturel \(n\).
\begin{enumerate}
  \item \(2^{4}\times 3^{2}\times 5^{6}\times 7^{2}=n^{2}\)
  \item \(2^{3}\times 3^{6}\times 5^{3}\times 7^{3}=n^{3}\)
  \item \(\left(\dfrac{4^{5}+4^{5}+4^{5}+4^{5}}{3^{5}+3^{5}+3^{5}}\right)
         \left(\dfrac{6^{5}+6^{5}+6^{5}+6^{5}+6^{5}+6^{5}}{2^{5}+2^{5}}\right)=2^{n}\)
  \item \(3^{2001}+3^{2002}+3^{2003}=n\times 3^{2001}\)
  \item \(8^{\,n}=2^{n}\times 2^{12}\)
\end{enumerate}

\bigskip

\subsection*{Ex 13 \; Calculs algébriques}
Soient \(a\) et \(b\) des nombres non nuls. Écrire les expressions sous la forme \(a^{n}\times b^{m}\) avec \(n\) et \(m\) entiers relatifs.

\medskip
\textbf{Série 1}
\[
\begin{aligned}
\mathrm{A}&=\dfrac{a^{2}b^{-3}}{a^{-2}b},
&\quad \mathrm{B}&=\dfrac{a^{6}b^{-4}}{a^{10}b^{-8}},
&\quad \mathrm{C}&=\dfrac{(a^{2}b)^{3}}{ba^{-2}},
&\quad \mathrm{D}&=\dfrac{(ab^{2})^{-1}}{(a^{2}b^{3})^{2}}
\end{aligned}
\]

\medskip
\textbf{Série 2}
\[
\begin{aligned}
\mathrm{A}&=a^{2}(ab)^{-3}(b^{-2})^{-3},
&\quad \mathrm{B}&=\dfrac{(ab^{2})^{-1}}{a^{-2}b^{-7}},\\
\mathrm{C}&=(a^{3}b)^{3}(a^{2}b^{5})^{5},
&\quad \mathrm{D}&=\dfrac{(ab^{3})^{-4}(a^{-2}b)^{2}}{a^{-6}b^{4}}
\end{aligned}
\]

\bigskip

\subsection*{Ex 14 \; Chiffres manquants}
Remplacer \(\,\bullet\,\) par des chiffres afin que les nombres obtenus vérifient la condition donnée. Écrire toutes les solutions possibles.
\begin{enumerate}
  \item \(5\ \bullet\ 8\ \bullet\ 2\) est divisible par \(9\).
  \item \(3\ \bullet\ 5\ \bullet\) est divisible par \(9\) et par \(2\).
  \item \(34\ \bullet\ 45\ \bullet\) est divisible à la fois par \(5\) et par \(9\).
  \item \(1\ \bullet\ 3\ \bullet\) est divisible par \(15\).
  \item \(\bullet\ 23\ 45\ \bullet\) est divisible par \(11\) et par \(3\).
\end{enumerate}

\bigskip

\subsection*{Ex 15 \; \textsc{pgcd} et \textsc{ppcm}}
On considère les nombres \(4\,116\) et \(2\,156\).
\begin{enumerate}
  \item Donner leur décomposition en facteurs premiers.
  \item Déterminer leur \(\mathrm{PGCD}\) et leur \(\mathrm{PPCM}\).
  \item Lequel de ces deux nombres a le plus de diviseurs ?
\end{enumerate}

\bigskip

\subsection*{Ex 16 \; Simplification de fraction}
En utilisant la décomposition en facteurs premiers, simplifier au maximum les fractions.
\[
\mathrm{A}=\dfrac{71\,610}{20\,790}
\qquad
\mathrm{B}=\dfrac{374\,850}{350\,350}
\qquad
\mathrm{C}=\dfrac{2\,635}{1\,274}
\qquad
\mathrm{D}=\dfrac{4\,923\,765}{980\,980}
\]

\bigskip

\subsection*{Ex 17}
Décomposer \(111\,111\) en produit de facteurs premiers.

\bigskip

\subsection*{Ex 18 \; Nombre de zéros}
Par combien de zéros se terminent les nombres suivants ?
\[
\begin{aligned}
\mathrm{A}&=1\times 2\times 3\times 4\times \cdots \times 9\times 10,\\
\mathrm{B}&=1\times 2\times 3\times 4\times \cdots \times 99\times 100,\\
\mathrm{C}&=100\times 101\times 102\times 103\times \cdots \times 998\times 999.
\end{aligned}
\]

\end{document}
